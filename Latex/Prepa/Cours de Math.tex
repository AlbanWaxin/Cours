\documentclass[12pt]{report}
\usepackage[utf8]{inputenc}
\usepackage[T1]{fontenc}
\usepackage{amsmath,amsfonts,amssymb,stmaryrd}
\usepackage[english]{babel}
\usepackage{ifthen}
\usepackage{pdfpages}

%=============Affichage=======================
\usepackage{fullpage}
\usepackage{mathtools}
\usepackage{lmodern}
\usepackage{xcolor}
\usepackage{enumitem}

\definecolor{almond}{rgb}{0.94, 0.87, 0.8}
\definecolor{dgreen}{rgb}{0.0, 0.5, 0.0}
\setlength{\topmargin}{-1.5cm}
\setlength{\textheight}{25cm}
\setlength{\textwidth}{16cm}
\setlength{\oddsidemargin}{-1.5cm}
\setlength{\evensidemargin}{50cm}

\newcommand{\rd}[1]{\textcolor{red}{#1}}
\newcommand{\g}[1]{\textcolor{lime}{#1}}
\newcommand{\dg}[1]{\textcolor{dgreen}{#1}}
\newcommand{\blue}[1]{\textcolor{blue}{#1}}
\newcommand{\cy}[1]{\textcolor{cyan}{#1}}
\newcommand{\blz}{$\blacklozenge$}
\newcommand{\ns}{\\\indent\indent\vspace{0.25cm}}
\newcommand{\df}{$\equiv$ }
\setcounter{secnumdepth}{5}% profondeur de la table des matières
\usepackage{titlesec}


\titleformat{\chapter}[frame]
{\Huge}
{\filright\rmfamily\bfseries\Huge\enspace\thechapter\enspace}
{18pt}
{\rmfamily\huge\bfseries\filcenter}
% rmfamily=roman, sffamily = sans serif ou ttfamily =type writer
\usepackage[many]{tcolorbox} % Creation de box collorable pour le texte non intégré
\newtcolorbox{mybox}{colback=white,
colframe=red,arc=0mm}

\newtcolorbox{demo}{colback=almond,
colframe=black,arc=0mm}

\renewcommand*{\overrightarrow}[1]{\vbox{\halign{##\cr
 \tiny\rightarrowfill\cr\noalign{\nointerlineskip\vskip1pt}
 $#1\mskip2mu$\cr}}}

\newcommand*{\Coordp}[4]{%
 \ensuremath{{#1}\,
   \left\lvert
     \begin{matrix}
       #2\\
       #3\\
       #4
     \end{matrix}
   \right.% Ne pas oublier le délimiteur invisible.
 }}
 \newcommand{\divp}[2]
	{
	  \frac{\partial #1}{\partial #2}
	}

\newcommand{\divt}[2]
	{
	  \frac{d #1}{d #2}
	}

\newcommand{\divpsnd}[2]
	{
	  \frac{\partial^2 #1}{\partial #2^2}
	}

\newcommand{\divts}[2]
	{
	  \frac{d^2 #1}{d #2^2}
	}

\newcommand{\rem}[1]
{
\cy{\underline{\blz Remarque: }#1}\vspace{0.5cm}
}

\newcommand{\props}[1]
{
\begin{mybox}
\textbf{\rd{\underline{\blz Théorème:} #1}}
\vspace{0.5cm}
\newline
}

\newcommand{\thetap}{\dot{\theta}}

\newcommand{\rot}[1]
{
\overrightarrow{rot}(\overrightarrow{#1})
}

\newcommand{\grad}[1]
{
\overrightarrow{grad}(\overrightarrow{#1})
}

\newcommand{\lapl}[1]
{
\Delta(\overrightarrow{#1})
}

\newcommand{\divs}[1]
{
div(\overrightarrow{#1})
}

\newcommand{\prope}
{
\end{mybox}
}

\newcommand{\defis}[1]
{
\begin{mybox}
\textbf{\rd{\underline{\blz Définition:} #1}}
\vspace{0.5cm}
\newline
}
\newcommand{\defie}
{
\end{mybox}
}
\newcommand{\demos}[1]
{
\begin{demo}
\textbf{\underline{\blz Démonstration:} #1}
\newline
}
\newcommand{\demoe}
{
\end{demo}
}
\newcommand{\exe}[1]
{
\underline{\blz Exemple:} #1
}

\newcommand{\vs}
{
\vspace{0.25cm}
}
%=============================================

%\usepackage[cm]{aeguill}

%=============Mathématiques=================

%--------------Raccourcis:------------------
\newcommand{\R}{\mathbb{R}}
\newcommand{\C}{\mathbb{C}}
\newcommand{\N}{\mathbb{N}}
\newcommand{\Q}{\mathbb{Q}}
\newcommand{\Z}{\mathbb{Z}}
\newcommand{\K}{\mathbb{K}}
\newcommand{\M}{\mathcal{M}}
\newcommand{\nint}[1]{#1 \in \N}
\newcommand{\zint}[1]{#1 \in \N^*}
\newcommand{\limi}[1]{\underset{#1 \to \infty}{lim}}
\newcommand{\limn}[2]{\underset{#1 \to #2}{lim}}
\newcommand{\x}{\times}
\newcommand{\un}[1]{u_{#1}}
\newcommand{\uns}{(u_n)_\nint{n}}
\newcommand{\Sn}[1]{S_{#1}}
\newcommand{\Sns}{(S_n)_\nint{n}}

\newcommand{\seriegu}{\sum u_n}
\newcommand{\seriegv}{\sum v_n}
\newcommand{\harmonique}{\sum \frac{1}{n}}
\newcommand{\SRieman}{\sum \frac{1}{n^\alpha}}
\newcommand{\serie}[3]{\sum_{#1}^{#2}{#3}}
\newcommand{\satps}{série à terme positif}
\newcommand{\satp}{séries à termes positifs}

\newcommand{\abs}[1]{\left\lvert#1\right\rvert}
\DeclarePairedDelimiter{\ceil}{\lceil}{\rceil}

%Format de fonctions:
\newcommand{\fct}[5]
	{
	  \begin{array}{ccccc}
		#1 & : & #2 & \to & #3 \\
	    && #4 & \mapsto & #5 \\
	  \end{array}
	}
\newcommand{\dfct}[2] {#1 \mapsto #2}
%===================TESTS===================

\begin{document}
\chapter{Series Numériques}

Dans tout le chapitre $\K$ = $\C$ ou $\R$

\section{Série numérique}

\defis{Série numérique}
Soit $(u_n)_{n\in\N}$ un suite d'éléments de $\K$ \\ On pose $\forall n \in \N , S_n = \sum_{k = 0}^n u_k$ \ns

On appelle \rd{série de terme général $u_n$}, notée $\sum u_n$, la suite $(S_n)_{n\in\N}$ \ns

$S_n$ est appelée \rd{Somme partiel d'ordre n}
\defie

\rem{}
\begin{enumerate}
  \item Il se peut que la suite $(u_n)_{n \geq n_0}$ ne soit définie qu'à partir d'un rang $n_o$. On pose alors $\forall n \geq n_0$ , $S_n = \sum_{k=n_0}^{n}u_k$
  \item On peut définir une structure d'espace vectoriel sur l'ensemble des séries en posant \ns
  \begin{itemize}
    \item $\sum u_n + \sum v_n = \sum (u_n +v_n)$
    \item $\lambda \sum u_n = \sum (\lambda u_n)$
  \end{itemize}
  \item \underline{Bijection Suite Série:} \rd{Quand on connait $u_n$} , on connait $S_n$ : $S_n = \sum_{k=0}^n u_k$\ns

  \dg{Réciperoque:} $\begin{cases} u_0 = S_0 \\
  \forall n \geq 1 , u_n = S_n -S_{n-1}
  \end{cases}$
\end{enumerate}

\newpage

\exe{}
\begin{itemize}
  \item Les séries géométrique: $\sum q^n$ \ns
On pose: $\forall n \in \N u_n = q^n$. \ns
On a alors $\forall n \in \N, S_n = \begin{cases} n+1, $si $ q=1 \\
\frac{1-q^{n+1}}{1-q}, $si $ q \neq 1
\end{cases}$
  \item Les séries de Riemann: \ns
$\sum \frac{1}{n^\alpha} , \alpha \in \R$ On ne connait pas $S_n$
\end{itemize}

\section{Séries Convergentes}

\defis{Série convergente, Somme d'une série}
Soit $\sum u_n$ une série d'élément de $\K$ \\
Soit $(S_n)_{n \in \N}$ la suite des sommes partielles \ns

On dit que $\sum u_n$ converge ssi la suite $(S_n)$ converge
et on note $S = \sum_{n=0}^{+\infty}$ \\ Sinon $\sum u_n$ diverge
\defie

\rem{}
  \begin{itemize}
    \item Ne pas écrire $\sum_{n=0}^{+\infty}u_n$ avant d'avoir montrer que $\sum u_n$ converge
    \item On peut modifier ou supprimer (troncature) un nombre fini de termes sans modifier la nature de $\sum u_n$
  \end{itemize}

\defis{Reste d'une série convergente}
  Soit $\sum u_{n+1}$ une série convergente de somme $S$ \ns
  On pose $\forall n \N, R_n = S - S_n = \sum_{k=n+1}^{+\infty}$

  $R_n$  est appelé reste d'ordre n de la série
\defie

\exe{Série géométrique:}\ns

$\sum q^n , q \in \R^*$\ns
On connait $S_n$,
$S_n= \begin{cases}
        n+1, $si $ q=1 \\
        \frac{1-q^{n+1}}{1-q}, $si $ q \neq 1
      \end{cases}$

Alors $\sum q^n$ converge ssi $ q \in [ 0,1 [$ \ns
On a alors \rd{$S=\frac{1}{1-q}$}

d'ou $R_n = \frac{1}{1-q} - \frac{1-q^{n+1}}{1-q} = \frac{q^{n+1}}{1-q}$

\newpage

\exe{Série harmonique} $\sum \frac{1}{n}$\ns

On pose $\forall n \in \N* , u_n = \frac{1}{n} et S_n = \sum_{k=1}^n \frac{1}{k}$

\demos{Divergence Série harmonique}
\begin{itemize}
  \item Méthode 1 : Par l'absurde
Supposons que $\harmonique$ converge\\
Soit $\zint{n}$\\
On a $S_{2n}-S_n = \serie{k=n+1}{2n}{\frac{1}{k}}$ \\
Donc $\forall \zint{n} , S_{2n}-S_n \geq n \times \frac{1}{2n} = \frac{1}{2}$ \\
or $\limi{n}(S_{2n}-S_n) = S-S = 0$\\
\rd{CONTRADICTION}

  \item Méthode 2 : Concavité de $\dfct{x}{ln(1+x)}$

Par concavité de $\dfct{x}{ln(1+x)}$ on a : $\forall x > -1, ln(1+x) \leq x$\\
En particulier $\zint{k} , ln(1+ \frac{1}{k}) \leq \frac{1}{k} $\ns

Ainsi, $\forall \zint{n}, \serie{k=1}{n}{ln(1+\frac{1}{k})} \leq S_n$\\
or $ln(1+\frac{1}{k}) = ln(1+k) - ln(k)$\\
Donc $\forall \zint{n}, Sn \geq ln(n+1)$\\
or$ \limi{n}ln(n+1) = +\infty$\\
Donc $\limi{n}S_n = +\infty$ , Donc $\harmonique$ diverge\\

  \item Méthode 3 : Par intégrales

Soit $\fct{f}{[1,+\infty[}{\R}{x}{\frac{1}{x}}$

Par décroissance de $f$ sur $[k,k+1]$ on a : $\frac{1}{k} \geq \int_k^{k+1}\frac{1}{x}dx$

En sommant $\forall \zint{n},S_n \geq \int_1^{n+1}\frac{1}{x}dx$\\
Donc $\limi{n}S_n = +\infty$
\end{itemize}
\demoe

\exe{Série harmonique alternée} : $\sum \frac{(-1)^{n-1}}{n}$

Soit $S_n  = \serie{k=1}{n}{\frac{(-1)^{n-1}}{n}}$

Montrons que $(S_{2n})_\zint{n} (S_{2n+1})_\zint{n}$ convergent et ont la même limite\\
On a si $\zint{n}$ \ns

$S_{2n} = \serie{k=1}{2n}{\frac{(-1)^{k-1}}{k}} \ns = \serie{j=1}{n}{\frac{1}{2j-1}}-\serie{j=1}{n}{\frac{1}{2j}} \ns = \serie{k=1}{2n}{\frac{1}{k}} - 2 \serie{k=1}{n}{\frac{1}{2k}}$

Donc $S_{2n} = \sum_{k = n+1}^{2n} \frac{1}{k}$

On identifie une série de Riemann de terme $\fct{f}{\R}{\R}{\frac{k}{n}}{\frac{1}{1+\frac{k}{n}}}$

d'ou $S_{2n} = \frac{1}{n} \serie{k=1}{n}{f(\frac{k}{n})}$\ns

Or f est continue\\
D'où $(S_{2n})_\zint{n}$ converge vers $\int_0^1 f(x)dx = ln(2)$\ns

\exe{Série Télescopique} $\sum \frac{1}{n(n+1)}$

Or $\forall \zint{n} , \frac{1}{n(n+1)} = \frac{1}{n} - \frac{1}{n+1}$

D'ou $\forall \zint{n}, S_n = \serie{k=1}{n}{\frac{1}{k}-\frac{1}{k+1}} $ \ns
$= 1 - \frac{1}{n+1}$

Donc $\seriegu$ converge, et $\serie{n=1}{+\infty}{\frac{1}{n(n+1)}} =1$

\props{Convergence des séries Télescopiques}
Soit $(u_n)_\nint{n}$ une suite numérique\\
La suite \rd{$(u_n)_\nint{n}$ converge} ssi la série Télescopique \rd{$\sum (u_{n+1} -u_n)$ converge}\\

Avec $S = \serie{n=0}{+\infty}{u_{n+1}-u_n} = \limi{n}u_n -u_0$
\prope

\demos{}
On a $\forall \nint{n}, S_n = \serie{k=0}{n}{u_{k+1}-u_k} = u_{n+1}-u_0$ par télescopage d'où le résultat
\demoe

\props{Condition de convergence}
$\seriegu$ converge $\Rightarrow (u_n)_\zint{n}$ converge vers 0 \\
Mais Réciperoquement $(u_n)_\zint{n}$ converge vers 0 $\nRightarrow \seriegu$ converge (cf $\harmonique$)
\prope

\demos{}
Soit $\zint{n}$\\
$u_n = S_n - S_{n-1}$\ns
Or $(S_n)_\zint{n}$ et $(S_{n-1})_\zint{n}$ convergent et ont la même limite\\
Donc $\uns$ converge vers 0
\demoe

\rem{Vocabulaire} si $\uns$ ne converge pas vers 0, on dit alors que $\seriegu$ diverge grossièrement

\props{Cas $\K = \C$}
Soit $\sum z_n$ une série complexe\\
On pose $\forall \nint{n}, z_n =x_n +iy_n$\\
or $\sum z_n $ converge ssi la suite $(\serie{k=0}{n}{z_k})_n$ converge\\
Ssi $\begin{cases}
\text{les suites réelles} (\serie{k=0}{n}{x_k})_n \text{et} (\serie{k=0}{n}{y_k})_n \text{convergent}\\
\text{les séries} \sum x_n \text{et} \sum y_n \text{convergent}
\end{cases}$
\prope

\props{Espace Vectoriel des séries convergentes}
Soit $\seriegu$ et $\seriegv$ deux séries convergentes\\
Soit $\lambda \in \R$

Alors $\sum \lambda u_n +v_n$ converge et $\serie{k=0}{+\infty}{\lambda u_k+ v_k} = \lambda \serie{k=0}{+\infty}{u_k} + \serie{k=0}{+\infty}{v_k}$
\prope

\demos{}
On a $\forall \nint{n} ,\serie{k=0}{n}{\lambda u_k+ v_k} = \lambda \serie{k=0}{n}{u_k} + \serie{k=0}{n}{v_k}$\ns

Comme $\seriegu$ et $\seriegv$ convergent les sommes partielles aussi
Donc on a le résultat attendu
\demoe

\section{Série de nombres réels positifs}

Comme la nature d'une série ne dépend pas des premiers termes et qu'elle est invariante par multiplication par -1,on pourra adapter les théorèmes de ce paragraphe au cas des séries de nombres réels de signe constant à partir d'un certain rang

\subsection{Théorème fondamental pour les séries à termes positifs}

\props{fondamental}
Soit $\seriegu$ une série de nombres réels positifs et $\Sns$ la suite de ses sommes partielles\\
$\seriegu$ converge ssi la suite $\Sns$ est majorée\\
Dans ce cas $S= sup(S_n)$ \ns Sinon $\limi{n} S_n = +\infty$
\prope

\demos{}
Montrons que $\Sns$ est croissante\\
Soit $\nint{n}$. On a $\Sn{n+1} - \Sn{n} = \un{n+1} \geq 0$ \\
Donc $\Sns$ est croissante\\
On conclue avec le Théorème des suites croissantes et majorées
\demoe

\subsection{Théorèmes de comparaison pour les \satps}

On va comparer la nature de deux séries à termes positifs au sens de $ "\leq","O","o",""\sim"$

\props{Majorations pours les \satps}
Soient $\seriegu$ et $\seriegv$ deux séries de nombres réels positifs
On suppose que
\begin{enumerate}[label=\roman*)]
  \item $\forall \nint{n} , u_n \leq v_n$
  \item $ \seriegv$ converge
\end{enumerate}
Alors $\seriegu$ converge et $\serie{n=0}{+\infty}{u_n} \leq \serie{n=0}{+\infty}{v_n}$
\prope

\demos{}
D'après i), par somme finie: \\
(*) $\forall \nint{n}, \serie{k=0}{n}{u_k} \leq \serie{k=0}{n}{v_k}$ \ns

Comme $\seriegv$ converge, la suite $(\serie{k=0}{n}{v_k})_\nint{n}$ converge \\
Donc majorée.

D'ou d'apres (*) la suite $(\serie{k=0}{n}{u_k})_\nint{n}$ est majorée\\
Par Théorème fondamental pour les \satps, $\seriegu$ converge et en passant à la limite dans (*) $\serie{n=0}{+\infty}{u_n} \leq \serie{n=0}{+\infty}{v_n}$
\demoe

\props{comparaison au sens de $"O"$ et $"o"$}
Soient $\seriegu$ et $\seriegv$  deux séries de nombres réels positifs\\
Supposons
\begin{enumerate}[label=\roman*)]
  \item $u_n = O(v_n) $ou$ u_n = o(v_n)$
  \item $\seriegv$ converge
\end{enumerate}
Alors $\seriegu$
\prope

\demos{}
Comme $ u_n = o(v_n) \Rightarrow u_n = O(v_n)$ on ne démontre que le second cas\\

Par définition de i), $\exists M \in \R^+,\exists N \in \N,\forall n geq N , \rd{0} \leq \un{n} \leq M v_n$ \\
Par ii) $\seriegv$ converge donc $sum M v_n$ \\
Par théorème de majoration pour les \satps, $\seriegu$ converge
\demoe

\props{comparaison au sens de $\sim$}
Soient $\seriegu$ et $\seriegv$  deux séries de nombres réels positifs\\
On suppose que $u_n \sim v_n$\\
Alors $\seriegu$ converge $\Leftrightarrow \seriegv$ converge
\prope

\demos{}
On a $u_n \sim v_n$\\
Donc  $u_n = O(v_n)$ et  $v_n = O(u_n)$
Par théorème précédent on obtient l'équivalence
\demoe

\props{Série de Riemann}
Soit $\alpha \in \R$ \\
$\SRieman$ converge $\Leftrightarrow \alpha > 1$
\prope

\demos{}
cas 1: $\alpha  \leq 1$\\
Or $\forall \zint{n}, \frac{1}{n} \leq 1$\\
D'ou avec $\alpha \leq 1$ on a $\forall \zint{n}, \frac{1}{n^\alpha} \geq \frac{1}{n} \geq 0$\\
Or $\harmonique$ diverge d'ou par théorème de majoration pour les \satps \ns

cas 2: $\alpha > 1$ Par intégrale\\
Soit $\fct{f}{[1,+\infty[}{\R^+}{x}{\frac{1}{x^\alpha}}$

On note $f$ décroissante et Par croissance de l'intégrale:\\
$\forall k \geq 2, \frac{1}{k^\alpha} \leq \int_{k-1}^k \frac{1}{x^\alpha} dx$\\
D'ou $\forall n \geq 2, \serie{k=1}{n}{\frac{1}{k^\alpha}} \leq \int_1^n \frac{1}{x^\alpha} dx$\ns

or $\int_1^n \frac{1}{x^\alpha} dx $\ns

$ = \int_1^n x^{-\alpha}$\ns

$= \left[\frac{x^{-\alpha +1}}{-\alpha +1} \right]_1^n$ \ns

$= \frac{n^{-\alpha+1}-1}{-\alpha +1}$

Donc $\int_1^n \frac{1}{x^\alpha}dx = \frac{(1-n^{-\alpha	+1})}{\alpha -1}$\ns

D'où $\forall n \geq 2, \serie{k=2}{n}{\frac{1}{k^\alpha}} \leq \frac{(1-n^{-\alpha	+1})}{\alpha -1} \leq \frac{1}{\alpha-1}$

Ainsi par théorème fondamental pour les \satps, $\SRieman$ converge\\
En pratique, on compare la nature de $\seriegu$ à celle de $\SRieman$ ou à celle de $\serie{}{}{q^n}$, en formant le quotient $(\frac{u_n}{\frac{1}{n^\alpha}}) = n^\alpha u_n$ ou le quotient $\frac{\un{n}}{q^n}$
\demoe

\exe{$\sum e^{-(ln(n))^a}$}

$\seriegu$ est une \satp \\
Si $a=1 \un{n} = \frac{1}{n}$ et $\seriegu$ diverge\\ \vskip{1cm}
Cas 1 : $a<1$\\
On a $\forall n \geq 3, ln(n) \geq 1$\\
Comme $a <1$, on $\forall n \geq 3, ln(n)^a \leq ln(n)$ \\
Donc
















\end{document}
