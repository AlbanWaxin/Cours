\documentclass{beamer}
\usetheme{Boadilla}
\usecolortheme{beaver}
\usepackage[utf8]{inputenc}
\usepackage[T1]{fontenc}
\usepackage{amsmath,amsfonts,amssymb,stmaryrd}
\usepackage[english]{babel}
\usepackage{minted}
\usepackage{xcolor}
\usepackage{color}
\usepackage{tcolorbox}
\tcbuselibrary{minted,skins}
%\usemintedstyle{monokai}
%\definecolor{bg}{rgb}{0.25,0.25,0.25}
%\newminted[ocaml2]{Ocaml}{linenos=true, texcl=true}
%\newminted[ocaml1]{Ocaml}{linenos=true, texcl=true, bgcolor=bg}
\usepackage{mathtools}
\usepackage[ruled,vlined]{algorithm2e}

\title{TIPE: Détection de Contours}
\subtitle{Détection de mouvement}
\author{Alban WAXIN n°39534}
\date



\newtcblisting{ocaml3}[1][]{
  listing engine=minted,
  colback=ocamlbg,
  colframe=black!70,
  listing only,
  minted style=monokai,
  minted language=Ocaml,
  minted options={linenos=true,numbersep=3mm,texcl=true,#1},
  left=5mm,enhanced,
  overlay={\begin{tcbclipinterior}\fill[black!25] (frame.south west)
            rectangle ([xshift=5mm]frame.north west);\end{tcbclipinterior}}
}
\definecolor{ocamlbg}{rgb}{0.25,0.25,0.25}

\renewcommand*{\overrightarrow}[1]{\vbox{\halign{##\cr
  \tiny\rightarrowfill\cr\noalign{\nointerlineskip\vskip1pt}
  $#1\mskip2mu$\cr}}}

\newcommand*{\Coordp}[4]{%
  \ensuremath{{#1}\,
    \left\lvert
      \begin{matrix}
        #2\\
        #3\\
        #4
      \end{matrix}
    \right.% Ne pas oublier le délimiteur invisible.
  }}

\newcommand{\divp}[2]
	{
	  \frac{\partial #1}{\partial #2}
	}

\newcommand{\divt}[2]
	{
	  \frac{d #1}{d #2}
	}

\newcommand{\divpsnd}[2]
	{
	  \frac{\partial^2 #1}{\partial #2^2}
	}

\newcommand{\divts}[2]
	{
	  \frac{d^2 #1}{d #2^2}
	}

\newcommand{\demos}[1]
{
\cy{\underline{\blz Démonstration: }#1}\vspace{0.5cm}
}

\newcommand{\props}[1]
{
 \begin{mybox}
 \textbf{\rd{\underline{\blz Propriétés:} #1}}
 \vspace{0.5cm}
 \newline
}

\newcommand{\thetap}{\dot{\theta}}

\newcommand{\rot}[1]
{
 \overrightarrow{rot}(\overrightarrow{#1})
}

\newcommand{\grad}[1]
{
 \overrightarrow{grad}(\overrightarrow{#1})
}

\newcommand{\lapl}[1]
{
 \Delta(\overrightarrow{#1})
}

\newcommand{\divs}[1]
{
 div(\overrightarrow{#1})
}

\newcommand{\prope}
{
 \end{mybox}
}

\newcommand{\defis}[1]
{
 \begin{mybox}
 \textbf{\rd{\underline{\blz Définition:} #1}}
 \vspace{0.5cm}
 \newline
}
\newcommand{\defie}
{
 \end{mybox}
}
\newcommand{\discu}[1]
{
 \blue{\underline{\blz Discussion:} #1}
}

\newcommand{\vs}
{
 \vspace{0.25cm}
}
\newcommand{\ns}{\\ \indent \indent \vspace{0.15cm}}
\newcommand{\MP}{\mathbb{P}}
\newcommand{\bxt}[1]{\boxed{\text{#1}}}
\newcommand{\rd}[1]{\textcolor{red}{#1}}
\newcommand{\R}{\mathbb{R}}
\newcommand{\C}{\mathbb{C}}
\newcommand{\N}{\mathbb{N}}
\newcommand{\Q}{\mathbb{Q}}
\newcommand{\Z}{\mathbb{Z}}
\newcommand{\K}{\mathbb{K}}
\newcommand{\M}{\mathcal{M}}

\newcommand{\abs}[1]{\left\lvert#1\right\rvert}
\DeclarePairedDelimiter{\ceil}{\lceil}{\rceil}
\newcommand{\fct}[5]
	{
	  \begin{array}{ccccc}
		#1 & : & #2 & \to & #3 \\
	    && #4 & \mapsto & #5 \\
	  \end{array}
	}

\begin{document}
\frame{\titlepage}

\begin{frame}
%\includegraphics{}
\begin{block}
{Problématique:} Comment identifier les contours d'un objet au sein d'un document image?
\end{block}
\end{frame}

\begin{frame}
\frametitle{Sommaire}
\begin{columns}[t]
  \begin{column}{7.5cm}
  \tableofcontents[sections={1,2}]
  \end{column}
  \begin{column}{4.5cm}
  \tableofcontents[sections={3}]
  \end{column}
  \end{columns}
\setcounter{tocdepth}{2}
\end{frame}

\section{Introduction}
\subsection{Définitions}

\begin{frame}[fragile]
\frametitle{Définitions}

Cadre d'étude: Traitement des images visant à l'extraction de données significatives.

\begin{itemize}

\item \underline{Image:} matrice de taille n$\times$m (nombre de pixels en largeur et en longueur).

\item \underline{Pixel:} unité de base de l'image codée par un entier entre 0 et $2^{24}$ ou un triplet d'entier(RGB) entre 0 et 255
 représentant l'intensité lumineuse (réfléchie et émise par le corps).

\item \underline{Contour/Bordure :} ligne séparant deux zones de contenu homogène connexes.
\begin{figure}[h]
    \begin{minipage}[c]{.46\linewidth}
        \centering
        \includegraphics[scale=0.5]{../../Downloads/TIPE (2)/contour brutal.jpg}
        \caption{Contour en escalier}
    \end{minipage}
\end{figure}
\end{itemize}
\end{frame}

\subsection{Photos d'étude}

\begin{frame}
\frametitle{Photos d'étude}
\begin{figure}[h]
    \begin{minipage}[c]{.46\linewidth}
        \centering
        \includegraphics[width=5cm,height=5cm,keepaspectratio]{../../Downloads/TIPE (2)/TIPE/cell.jpg}
        \caption{Cell.jpg}
    \end{minipage}
    \hfill%
    \begin{minipage}[c]{.46\linewidth}
        \centering
        \includegraphics[width=5cm,height=5cm,keepaspectratio]{../../Downloads/TIPE (2)/TIPE/Moon.jpg}
        \caption{Moon.jpg}
    \end{minipage}
\end{figure}

\hspace{2cm}

\end{frame}

\begin{frame}
\section{Détection des contours}
\begin{block}
{ } \center Détection des contours
\end{block}
\end{frame}

\subsection{Définition de contour dans le domaine continu ordre 1 et 2}
\begin{frame}
\frametitle{Définition de contour dans le domaine continu ordre 1}

\underline{Contour :} point de haute variation des valeurs de la fonction image.

\vspace{0.25cm}
De part le caractère bidimensionnel des images on assimile la valeur d'un pixel à une fonction $\fct {I}{\R ^2}{\R}{(x,y)}{i}$.

\begin{itemize}

\item Ainsi les variations de l'image sont représentées par les maximas du gradient de la fonction $I$ , $\vec{\nabla} I= \divp{I(x,y)}{x} \vec{x} , \divp{I(x,y)}{y} \vec{y}$.
\item Direction :
$\vec{d} = \frac{\overrightarrow{\nabla f}}{||\overrightarrow{\nabla f}||}$.
\item On a alors $\mathbb{P} = \lbrace(x,y) \in \R^2 | max\lbrace\vert\vert \nabla I \vert\vert. \vec{d}\rbrace\rbrace  $.
\ns

$\vert\vert \nabla I \vert\vert. \vec{d}$ que l'on notera aussi $\divp{I}{\vec{d}}$.
\end{itemize}
\end{frame}

\begin{frame}
\frametitle{Définition de contour dans le domaine continu ordre 2}
Un maximum de variation est aussi atteint pour un passage à zéro de la dérivée seconde.
\begin{itemize}

\item <1-> Ce qui donne $\mathbb{P} = \lbrace(x,y) \in \Z^2 | \divpsnd{I}{\vec{d}} = 0 \rbrace  $.
\item <2> Or par projection de $\vec{d}$ sur $\vec{x} $ et $\vec{y}$
$\vec{d} = \cos\theta \vec{x} + \sin\theta \vec{y}$

\vspace{0.15cm}
Il suit par opérations sur les dérivées partielles:

$\divp{I}{\vec{d}} = \divp{I}{x} \cos\theta + \divp{I}{y}\sin \theta$

et ainsi par théorème de Schwartz:

$\divpsnd{I}{\vec{d}} = \divpsnd{I}{x} \cos^2 \theta + 2 \frac{\partial^2 I}{\partial x \partial y} \cos\theta \sin \theta + \divpsnd{I}{y} \sin^2\theta$
\item <3> Or par définition $\Delta I = \divpsnd{I}{x} +\divpsnd{I}{y}$ donc pour de faibles valeurs de $\theta$ on peut approximer $\divpsnd{I}{\vec{d}} = \Delta I$.
\item <3-> D'où $\mathbb{P} = \lbrace(x,y) \in \Z^2 ; \Delta I = 0 \rbrace$.

\end{itemize}
\end{frame}

\subsection{Modélisation applicable au modèle discret}
\begin{frame}
\frametitle{Modélisation applicable au modèle discret}
Pour une mise en pratique efficace on utilise des applications linéaires $\fct{Tr}{(\Z^2)^k}{\Z}{x_0,...,x_k}{\sum_0^k a_i x_i} $

\begin{itemize}
\item  Soit $x_i$ pixels voisins du pixel courant, on applique une \underline{convolution (noté $*$)} d'un masque.

\item Formule de convolution : $f * g$ = $\sum_{i=0}^k f(k-i)g(i)$.

\item \includegraphics[scale=0.5]{../../Downloads/TIPE (2)/masquage.jpg}

\item le pixel central devient après traitement $Tr(x5) = x_9 a_1 + x_8 a_2 + x_7 a_3 + x_6 a_4 + x_5 a_5 + x_4 a_6 +x_3 a_7 +x_2 a_8 +x_1 a_9$.
\end{itemize}
\end{frame}

\subsection{Discrétisation et masques du Gradient}
\begin{frame}
\frametitle{Discrétisation et masque du Gradient}
\begin{itemize}
\item <1->Rappel $\vec{\nabla} I = \divp{I}{x} + \divp{I}{y}$.

\item <1-> Par nature la fonction $I$ est $C^4$ d'où par Théorème de Taylor$-$Young on a en $a$:

Soit $y \in R$.

$\divp{I}{x}(x,y) = I(a,y) + \divp{I}{x}(a,y)(x-a) + \frac{\divpsnd{I}{x}(a,y)(x-a)^2}{2} + \frac{\frac{d^3I(x,y)}{dx^3}(a,y)(x-a)^3}{6} + O((x-a)^4)$.

\item <2-> En posant $a = x$ on obtient(différentiabilité).

$\divp{I}{x}(x+h,y) = I(x,y) + \divp{I}{x}(x,y)(h) + \frac{\divp{I}{x}(x,y)(h)^2}{2} + \frac{\frac{d^3I(x,y)}{dx^3}(x,y)(h)^3}{6} + O(h^4)$ (1)

$\divp{I}{x}(x-h,y) = I(x,y) - \divp{I}{x}(x,y)(h) + \frac{\divp{I}{x}(x,y)(h)^2}{2} - \frac{\frac{d^3I(x,y)}{dx^3}(x,y)(h)^3}{6} + O(h^4)$ (2)
\end{itemize}
\end{frame}

\begin{frame}
\frametitle{Discrétisation et masque du Gradient}
En combinant (1) et (2) on obtient 3 valeurs possibles de $\divp{I}{x}(x)$:

\begin{itemize}

\item Approximation droite : $\divp{I}{x}(x,y) = \frac{\divp{I}{x}(x+h,y)-\divp{I}{x}(x,y)}{h} + O(h)$ à partir de (1).

\item Approximation gauche : $\divp{I}{x}(x,y) = \frac{\divp{I}{x}(x,y)-\divp{I}{x}(x,y)}{h} + O(h)$ à partir de (2).

\item Approximation centrée : $\divp{I}{x}(x,y) = \frac{\divp{I}{x}(x+h,y)-
\divp{I}{x}(x-h,y)}{2*h}$.


\item Donc les masques Gradients selon les deux directions sont

\begin{figure}[h]
    \begin{minipage}[c]{.46\linewidth}
        \centering
        \includegraphics[scale=0.35]{../../Downloads/TIPE (2)/gradientx.jpg}
        \caption{Masque Gradient x}
    \end{minipage}
    \hfill%
    \begin{minipage}[c]{.46\linewidth}
        \centering
        \includegraphics[scale=0.35]{../../Downloads/TIPE (2)/gradienty.jpg}
        \caption{Masque Gradient y}
    \end{minipage}
\end{figure}

\end{itemize}
\end{frame}


\subsection{Discrétisation et masque du Laplacien}
\begin{frame}
\frametitle{Discrétisation et masque du Laplacien}
\begin{itemize}
\item <1->Rappel $\Delta I = \divpsnd{I}{x} + \divpsnd{I}{y}$.

\item <1-> En réutilisant (1) et (2) on a:

(1)+(2) $->$ $\divpsnd{I}{x}(x,y) = \frac{\divp{I}{x}(x+h,y)+\divp{I}{x}(x-h,y)-2\divp{I}{x}(x,y)}{h^2} + O(h^2)$.


\item <2-> $\Delta I = \divpsnd{I}{x} + \divpsnd{I}{y}$

$=I(x+1,y) -2I(x,y) + I(x-1,y) + I(x,y-1) -2I(x,y) + I(x,y+1) $

$= I(x+1,y) + I(x-1,y) + I(x,y+1) - 4I(x,y) + I(x,y+1)$

\item <3-> Donc le masque Laplacien est un masque $3 \times 3$.
\begin{figure}
\includegraphics[scale=0.75]{../../Downloads/TIPE (2)/laplacien4.jpg}
\end{figure}
\end{itemize}
\end{frame}

\subsection{Limites du masque Laplacien}

\begin{frame}
\frametitle{Limites du masque Laplacien}

Le Laplacien souffre principalement de deux limites:

\begin{enumerate}
\item Une limite de courbure des contours
\item Une hypersensibilité au bruit
\end{enumerate}
\end{frame}

\begin{frame}
\frametitle{Résolution 1 :}
\begin{itemize}

\item Prise en compte des contours diagonaux: rotation de $\frac{\pi}{4}$ du masque précédent.
\begin{figure}
\includegraphics[scale=0.45]{../../Downloads/TIPE (2)/laplacien4_2.jpg}
\end{figure}

\item Ainsi en les combinant on obtient:

\begin{figure}
\includegraphics[scale=0.75]{../../Downloads/TIPE (2)/laplacien8.jpg}
\end{figure}
\end{itemize}
\end{frame}

\begin{frame}
\frametitle{Limites du masque Laplacien}

Résolution du problème du bruit du Laplacien: dilatation .
\begin{figure}
\includegraphics[scale=0.25]{../../Downloads/TIPE (2)/flou.jpg}
\end{figure}

On remarque que le nouveau filtre défini précédemment effectue aussi ce genre d'opérations.

\begin{figure}
\includegraphics[scale=0.35]{../../Downloads/TIPE (2)/calcul.jpg}
\end{figure}

Ainsi on obtient que Contours = |Initial - Flou|.

\end{frame}

\subsection{Présentation masques usuels}
\begin{frame}
\frametitle{Présentation masques usuels}

D'autres masques sont aussi utilisés pour le traitement des contours comme:


\begin{itemize}
\item Le filtre de Sobel:

\includegraphics[scale=0.35]{../../Downloads/TIPE (2)/sobx.jpg} \includegraphics[scale=0.35]{../../Downloads/TIPE (2)/soby.jpg}

\item Le filtre de Prewitt:

\includegraphics[scale=0.35]{../../Downloads/TIPE (2)/prewitx.jpg} \includegraphics[scale=0.35]{../../Downloads/TIPE (2)/prewity.jpg}


\item Le filtre de Kirsch:

\includegraphics[scale=0.35]{../../Downloads/TIPE (2)/kirsch.jpg}

\end{itemize}
\end{frame}

\subsection{Application}
\begin{frame}[fragile]
\frametitle{Application des filtres}

Les filtres précédents approximant le gradient on s'intéresse au maximum de leur module dans les directions.

Soit $It_x$ la matrice image après transformation selon l'axe x.

On considère alors $It = \sqrt{It_x^2 + It_y^2}$
\begin{center}
    \scalebox{0.75}{

    \begin{minipage}{\linewidth}
\begin{algorithm}[H]
\SetAlgoLined
	\KwData{M(matrice image),F(matrice filtre)}
	\KwResult{$It$ Matrice Traitée}
	\BlankLine
	$It \leftarrow$ MatriceVide()
	\For{i (paire d'entier) $\in$ M.size}
	{
	 $It.(i) \leftarrow$ Convolution(Vois(i) F)
	}

\caption{Algorithme d'application des filtres}
\end{algorithm}
Avec Voisin(i) et Convolution deux primitives qui récupèrent le voisinage du pixel i et effectuent le produit de convolution.
\end{minipage}%

    }
\end{center}
\end{frame}

\begin{frame}
\frametitle{Résultats intermédiaires}
\begin{center}
\includegraphics[scale=5]{../../Downloads/TIPE (2)/TIPE/pixel.png} \includegraphics[scale=5]{../../Downloads/TIPE (2)/TIPE/sobx.png}
\end{center}

On doit alors échelonner les valeurs, on ajoute donc ce test à la boucle de l'algorithme précédent.
\begin{center}
    \scalebox{0.75}{

    \begin{minipage}{\linewidth}
\begin{algorithm}[H]
\SetAlgoLined
	 \If {$It.(i) < 0$} {$It.(i) <- 0$\;}
	 \If {$It.(i) >16777215$} {$It.(i) <- 16777215$\;}

\caption{Algorithme d'application des filtres}
\end{algorithm}
\end{minipage}%

    }
\end{center}

\end{frame}

\subsection{Seuillage}
\begin{frame}[fragile]
\frametitle{Seuillage}
Pour éliminer une partie des pixels détectés comme contours mais non significatifs on peut appliquer un seuil.

\begin{center}
    \scalebox{0.75}{

    \begin{minipage}{\linewidth}
\begin{algorithm}[H]
\SetAlgoLined
	\KwData{M(matrice image),F(matrice filtre),X(seuil)}
	\KwResult{$It$ (Matrice Traitée)}
	\BlankLine
	$It \leftarrow$ MatriceVide()
	\For{i (paire d'entier) $\in$ M.size}
	{
	 \eIf {$It.(i) > X$} {$It.(i)<- 16777215$\;} {$It.(i)<- 0$}
	}

\caption{Algorithme de seuillage}
\end{algorithm}
\end{minipage}%

    }
\end{center}

\end{frame}

\begin{frame}[fragile]
\frametitle{Seuillage}


\begin{center}
    \scalebox{0.75}{

    \begin{minipage}{\linewidth}
\begin{algorithm}[H]
\SetAlgoLined
	\KwData{M(matrice image),F(matrice filtre), H(seuil haut),D(seuil bas)}
	\KwResult{$It$ Matrice Traitée}
	\BlankLine
	$It \leftarrow$ MatriceVide()
	\For{i (paire d'entier) $\in$ M.size}
	{
	 \If {$It.(i) > H$} {$It.(i)<- 16777215$\;}
	 \eIf {$It.(i) < B$} {$It.(i)<- 0$}
	{\If {Voisinsup(i)}{$It.(i)<- 0$}
	 \If {Voisininf(i)}{$It.(i)<- 16777215$}
	}
	}

\caption{Algorithme de seuillage par Hystérésie}
\end{algorithm}

Avec Voisininf et Voisinsup deux primitives qui vérifient si un voisin est inférieur/supérieur à un des seuils.
\end{minipage}%

    }
\end{center}
\end{frame}

\subsection{Résultats}
\begin{frame}
\frametitle{Quelques résultats:}

Dilatation:

\includegraphics[scale=0.2]{../../Downloads/TIPE (2)/TIPE/blur255.jpg}
\includegraphics[scale=0.25]{../../Downloads/TIPE (2)/TIPE/Moon.jpg}

\hspace{2.5 cm } Dilatée \hspace{4cm} Originale
\end{frame}

\begin{frame}
\frametitle{Quelques résultats:}

Sobel:

\includegraphics[scale=0.2]{../../Downloads/TIPE (2)/TIPE/sobelnoseuil.png}
\includegraphics[scale=0.2]{../../Downloads/TIPE (2)/TIPE/yessss59.png}

\hspace{2.5 cm } Sobel non seuillé \hspace{4cm} Seuil fixe
\end{frame}

\begin{frame}
\frametitle{Quelques résultats:}

Laplacien:


\begin{figure}
        \centering
        \includegraphics[scale=0.35]{../../Downloads/TIPE (2)/TIPE/cell_lap.jpg}
        \caption{Laplacien}
\end{figure}
\end{frame}

\begin{frame}
\frametitle{Quelques résultats:}
\begin{figure}[h]
    \begin{minipage}[c]{.46\linewidth}
        \centering
        \includegraphics[scale=0.2]{../../Downloads/TIPE (2)/TIPE/cell_laph.jpg}
        \caption{Seuillée par Hystérésie}
    \end{minipage}
    \hfill%
    \begin{minipage}[c]{.46\linewidth}
        \centering
        \includegraphics[scale=0.2]{../../Downloads/TIPE (2)/TIPE/cell_lap_s1.jpg}
        \caption{Seuillée manuellement}
    \end{minipage}
\end{figure}
\end{frame}

\begin{frame}
\frametitle{Synthèse résultats}

\center \includegraphics[scale=0.5]{../../Downloads/TIPE (2)/synthese.jpg}
\end{frame}

\begin{frame}
\frametitle{Illustration Synthèse: Sobel}

\begin{figure}[h]
    \begin{minipage}[c]{.46\linewidth}
        \centering
        \includegraphics[scale=0.2]{../../Downloads/TIPE (2)/TIPE/flou1.png}
        \caption{Bruit}
    \end{minipage}
    \hfill%
    \begin{minipage}[c]{.46\linewidth}
        \centering
        \includegraphics[scale=0.2]{../../Downloads/TIPE (2)/TIPE/cell_sobel_70.png}
        \caption{Contours épais}
    \end{minipage}
\end{figure}

\end{frame}

\begin{frame}
\frametitle{Illustration Synthèse: Laplacien}

\begin{figure}[h]
    \begin{minipage}[c]{.46\linewidth}
        \centering
        \includegraphics[scale=0.2]{../../Downloads/TIPE (2)/TIPE/moon_lap.png}
        \caption{Sursensibilité}
    \end{minipage}
    \hfill%
    \begin{minipage}[c]{.46\linewidth}
        \centering
        \includegraphics[scale=0.2]{../../Downloads/TIPE (2)/TIPE/lap230.png}
        \caption{Perte d'information par seuillage}
    \end{minipage}
\end{figure}

\end{frame}

\begin{frame}
\frametitle{Une amélioration possible le lissage}

\begin{figure}[h]
    \begin{minipage}[c]{.46\linewidth}
        \centering
        \includegraphics[scale=0.2]{../../Downloads/TIPE (2)/TIPE/10cell_lap_2blur_s.jpg}
        \caption{seuillage faible sur un double lissage moyen du Laplacien}
    \end{minipage}
    \hfill%
    \begin{minipage}[c]{.46\linewidth}
        \centering
        \includegraphics[scale=0.2]{../../Downloads/TIPE (2)/TIPE/60cell_sobel_2blur.jpg}
        \caption{Seuillage faible sur un double lissage moyen du sobel}
    \end{minipage}
\end{figure}
\end{frame}


\begin{frame}
\section{Traitements des contours}

\begin{block}
{ } \center Traitements des contours
\end{block}
\end{frame}


\begin{frame}
\frametitle{Présentation}
Afin que les informations sur les contours soient utilisables ceux-ci doivent être fermés et fins.

Les résultats fournis par le Laplacien donnent des contours fins mais ouverts.

Une opération de chainage en quatre temps permet de corriger cela :

\begin{enumerate}
\item Création de chaines
\item Filiation des chaines
\item Élimination
\item  Fusion
\end{enumerate}
\end{frame}

\begin{frame}
\frametitle {Image d'illustration du principe}
\begin{figure}[h]
    \begin{minipage}[c]{.46\linewidth}
        \centering
        \includegraphics[scale=0.2]{../../Downloads/TIPE (2)/Chainée1.png}
        \caption{Image contour}
    \end{minipage}
    \hfill%
    \begin{minipage}[c]{.46\linewidth}
        \centering
        \includegraphics[scale=0.2]{../../Downloads/TIPE (2)/Chainée.png}
        \caption{Résultat du Chainage}
    \end{minipage}
\end{figure}
\end{frame}

\subsection{Préparatifs}

\begin{frame}
\frametitle{Préparatifs}

Pour pouvoir mettre en place l'algorithme de chainage nous devons disposer de différentes informations au sein de classes.
\begin{itemize}
\item <1> les chaines seront composées de :
\begin{itemize}
\item une entête de classe entetes
\item de maillons de classe list maillon
\end{itemize}
\item <2> les entêtes seront composées de:
\begin{itemize}
\item un indice de classe entier
\item une tête de classe doublet d'entier
\item une queue de classe doublet d'entier
\item d'une famille de tête de classe liste d'entier et de char
\item d'une famille de queue de classe liste d'entier et de char
\end{itemize}

\item <2>les maillons sont des doublets d'entiers.
\end{itemize}
\end{frame}

\begin{frame}
\frametitle{Préparatifs}

On définit le voisinage d'un passé comme composé d' un passé d'un présent et d'un futur.
\begin{center}
\includegraphics[scale=0.45]{../../Downloads/TIPE (2)/format.jpg}
\end{center}
Une chaine est dite ouverte si l'on peut y rajouter des maillons sinon elle est fermée.

Il y a jonction entre chaine s'il y a deux chaines ouvertes non connexes au pixel courant et  une future ou inversement.
\end{frame}

\subsection{Étape 1: Création}
\begin{frame}
\frametitle{Création}

Pour pouvoir créer les chaines de contours on doit connaitre le nombre de chaines ouvertes/fermées dans le passé (NCO NCF) et le nombre de pixels contour dans le futur(NPF) du pixel courant.
\end{frame}

\begin{frame}[fragile]
\begin{center}
    \scalebox{0.7}{

    \begin{minipage}{\linewidth}
\begin{algorithm}[H]
\SetAlgoLined
	\KwData{M(matrice contour)}
	\KwResult{$It$ Matrice Traitée}
	\BlankLine
	$It \leftarrow$ MatriceVide()
	\For{i (paire d'entier) $\in$ M.size}
	{
	 NCO,NCF $\leftarrow$ CalculNCONCF(i) \;
	 NPF $\leftarrow$ CalculNPF(i)\;
	 \Case {NCO = 0}{Creerchaine(i)}
	 \Case {NCO = 1}{Ajoutchaine(i,num)
		\eIf {NCF >= 1}{FermeChaine(num)}{\If {NPF >= 2}{FermeChaine(num)}}}
	 \Case {NCO = 2} {Ajoutchaine(i,num1)\; Fermechaine(num1)\;Fermechaine(num2)\;}
	}
	FermetureChaines\;

\caption{Algorithme de création de chaines}
\end{algorithm}
\end{minipage}%

    }
\end{center}
\end{frame}

\begin{frame}
\frametitle{Illustrations des cas}
\begin{figure}[h]
    \begin{minipage}[c]{.46\linewidth}
        \centering
        \includegraphics[scale=0.4]{../../Downloads/TIPE (2)/o_0nco.jpg}
        \caption{Ouverture de chaine}
    \end{minipage}
    \hfill%
    \begin{minipage}[c]{.46\linewidth}
        \centering
        \includegraphics[scale=0.4]{../../Downloads/TIPE (2)/f_2npf_1nco.jpg}
        \caption{Ajout et Fermeture pour cause de 2 futures}
    \end{minipage}
	\hfill%
    \begin{minipage}[c]{.46\linewidth}
        \centering
        \includegraphics[scale=0.35]{../../Downloads/TIPE (2)/f_1nco_1ncf.jpg}
        \caption{Ajout et Fermeture pour cause de 1 passé ouvert et 1 passé fermé}
    \end{minipage}
\hfill%
    \begin{minipage}[c]{.46\linewidth}
        \centering
        \includegraphics[scale=0.35]{../../Downloads/TIPE (2)/f_2nco.jpg}
        \caption{Double Fermeture pour cause de deux passés ouverts}
    \end{minipage}

\end{figure}
\end{frame}

\begin{frame}
\frametitle{Ajustement de l'efficacité}
La méthode précédente de création de chaine n'est pas optimale car on considère une jonction dès qu'il y a deux chaines passées ouvertes or on ne considère que les chaines non connexes.



Ainsi en posant le voisinage du pixel comme :

\begin{center}
\includegraphics[scale=0.5]{../../Downloads/TIPE (2)/format2.jpg}
\end{center}


\begin{center}
    \scalebox{0.75}{

    \begin{minipage}{\linewidth}
\begin{algorithm}[H]
\SetAlgoLined
	\If {(Ouvert(P1) et Ouvert(P3)) ou (Ouvert(P3) et Ouvert(P4))}{}

\caption{Ajustements}
\end{algorithm}
\end{minipage}%

    }
\end{center}
\end{frame}
\subsection{Étape 2 : Affiliation}
\begin{frame}[fragile]
\frametitle{Affiliation}

Avant fusion des chaines il est nécessaire de les lier entre elles.

2 méthodes:
\begin{itemize}
\item Filiation post-chainage
\item Affiliation pendant le parcours grâce à une machine d'états

On utilisera la première méthode avec l'algorithme suivant:
\end{itemize}
\begin{center}
    \scalebox{0.75}{

    \begin{minipage}{\linewidth}
\begin{algorithm}[H]
\SetAlgoLined
	\KwData{Chainelist}
	\KwResult{Chainelist (complétée avec les liens de filiation)}
	\BlankLine
	\For{i (chaine) $\in$ Chaineliste}
	{
	 filiationQueue(i,Chainelist)\;
	 filiationTete(i,Chainelist)\;
	}


\caption{Algorithme de Filiation}
\end{algorithm}
\end{minipage}%

    }
\end{center}
Avec filiationQueue(i,Chainelist) filiationTete(i,Chainelist) deux primitives détaillées ci-après.
\end{frame}

\begin{frame}[fragile]
\frametitle{Affiliation}

\begin{center}
    \scalebox{0.75}{

    \begin{minipage}{\linewidth}
\begin{algorithm}[H]
\SetAlgoLined
	\KwData{Chaine}
	\KwResult{Chaine (complétée avec les liens de filiation)}
	\BlankLine
	tete $\leftarrow$ Chaine.head \;
	$v_t \leftarrow$ Passe(tete) \;
	\For{i (pixel) $\in$ $v_t$}
	{
	 \If {i.chaine != Chaine}
	 {
	  \eIf {Estconnexe(i,$v_t$)}
	  	{\If {Voisindirect(i,tete)}
	  	{Ajoutlientete(Chaine,i.chaine, i.type)}
	  	}{Ajoutlientete(Chaine,i.chaine, i.type)}}
	}


\caption{Algorithme de Filiation}
\end{algorithm}
\end{minipage}%

    }
\end{center}
Avec Estconnexe renvoie si le pixel est connexe à un autre passé et Voisindirect qui renvoie si le pixel considéré est directement connexe au pixel courant.
\end{frame}

\subsection{Étape 4: Élimination}
\begin{frame}[fragile]
\frametitle{Élimination}
On peut éliminer sur plusieurs points, on choisit la connectivité et la longueur.

\begin{center}
    \scalebox{0.5}{

    \begin{minipage}{\linewidth}
\begin{algorithm}[H]
\SetAlgoLined
	\KwData{ChaineList,seuil}
	\KwResult{ChaineList (mise à jour)}
	\BlankLine
	\For{i (chaine) $\in$ ChaineList}
	{
	 \If {deconnectee(i) and (i.maillons).size > 0}{supprimer i}
	 \If {(i.headlink).size = 0 and (i.taillink).size > 0 and Connection(i) and (i.maillons).size<seuil}{separer(i,i.taillink)\;reafilier(i.taillink)\;supprimer i}
	 \If {(i.headlink).size > 0 and (i.taillink).size = 0 and Connection(i) and (i.maillons).size<seuil}{separer(i,i.headlink)\;reafilier(i.headlink)\;supprimer i}
	 \If {Autoconnect(i) and( i.maillons).size<seuil }{separer(i,i.taillink)\;separer(i,i.headlink)\;reafilier(i.taillink)\;reafilier(i.headlink)\;supprimer i}
	}


\caption{Algorithme de Suppression}
\end{algorithm}
\end{minipage}%

    }
\end{center}

\end{frame}

\begin{frame}[fragile]
\frametitle{Complexité}
L'étape 1 de chaînage consiste en un balayage.

Sur chaque pixel on effectue :
\begin{itemize}
\item test de significativité: 1
\item test sur les passés:
\begin{itemize}
\item Type: 1
\item Récupération chaine, tête et queue : 2
\item État de la chaîne: 1

\end{itemize}
\item ajout à une chaine : 2
\end{itemize}

Cout total: 18 opérations * n $\times$ m

L'étape 2 de filiation parcours les chaines.

Sur chaque chaine(passé des extrémités) on effectue:
\begin{itemize}
\item vérification numéro de chaîne : 1
\item test de connexité avec autre voisin : 4*2
\item test de connexité direct au pixel courant : 1
\item ajout du lien : 1
\end{itemize}

Cout total:(11*2)*2* nombre de chaine

\end{frame}

\begin{frame}

\begin{figure}[h]
    \begin{minipage}[c]{.46\linewidth}
        \centering
        \includegraphics[scale=0.2]{../../Downloads/TIPE (2)/Chainée1.png}
        \caption{Image contour initial}
    \end{minipage}
    \hfill%
    \begin{minipage}[c]{.46\linewidth}
        \centering
        \includegraphics[scale=0.2]{../../Downloads/TIPE (2)/chainnage.png}
        \caption{Image contour avec chainage ajusté et fusionné}
    \end{minipage}
    \begin{minipage}[c]{.46\linewidth}
        \centering
        \includegraphics[scale=0.2]{../../Downloads/TIPE (2)/elimination.png}
        \caption{Image avec chaines éliminées}
    \end{minipage}
    \hfill
    \begin{minipage}[c]{.46\linewidth}
        \centering
        \includegraphics[scale=0.2]{../../Downloads/TIPE (2)/Chainée.png}
        \caption{Image chainée}
    \end{minipage}

\end{figure}
\end{frame}

\section{Annexe}

\begin{frame}[fragile]
\frametitle{Annexe: Code détection de contours}
\begin{figure}[h]
    \begin{minipage}[c]{.46\linewidth}
        \centering
        \includegraphics[scale=0.55]{../../Downloads/TIPE (2)/image code/sccontour1.jpg}

    \end{minipage}
\end{figure}
\end{frame}

\begin{frame}[fragile]
\frametitle{Annexe: Code détection de contours}
\begin{figure}[h]
    \begin{minipage}[c]{.46\linewidth}
        \centering
        \includegraphics[scale=0.55]{../../Downloads/TIPE (2)/image code/sccontour2.jpg}

    \end{minipage}
\end{figure}
\end{frame}

\begin{frame}[fragile]
\frametitle{Annexe: Code détection de contours}
\begin{figure}[h]
    \begin{minipage}[c]{.46\linewidth}
        \centering
        \includegraphics[scale=0.55]{../../Downloads/TIPE (2)/image code/sccontour3.jpg}

    \end{minipage}
\end{figure}
\end{frame}

\begin{frame}[fragile]
\frametitle{Annexe: Code détection de contours}
\begin{figure}[h]
    \begin{minipage}[c]{.46\linewidth}
        \centering
        \includegraphics[scale=0.55]{../../Downloads/TIPE (2)/image code/sccontour4.jpg}

    \end{minipage}
\end{figure}
\end{frame}

\begin{frame}[fragile]
\frametitle{Annexe: Code détection de contours}
\begin{figure}[h]
    \begin{minipage}[c]{.46\linewidth}
        \centering
        \includegraphics[scale=0.55]{../../Downloads/TIPE (2)/image code/sccontour5.jpg}

    \end{minipage}
\end{figure}
\end{frame}

\begin{frame}[fragile]
\frametitle{Annexe: Code détection de contours}
\begin{figure}[h]
    \begin{minipage}[c]{.46\linewidth}
        \centering
        \includegraphics[scale=0.55]{../../Downloads/TIPE (2)/image code/sccontour6.jpg}

    \end{minipage}
\end{figure}
\end{frame}

\begin{frame}[fragile]
\frametitle{Annexe: Code détection de contours}
\begin{figure}[h]
    \begin{minipage}[c]{.46\linewidth}
        \centering
        \includegraphics[scale=0.45]{../../Downloads/TIPE (2)/image code/sccontour7.jpg}

    \end{minipage}
\end{figure}
\end{frame}

\begin{frame}[fragile]
\frametitle{Annexe: Code chaînage}
\begin{figure}[h]
    \begin{minipage}[c]{.46\linewidth}
        \centering
        \includegraphics[scale=0.45]{../../Downloads/TIPE (2)/image code/chainage1.jpg}

    \end{minipage}
\end{figure}
\end{frame}


\begin{frame}[fragile]
\frametitle{Annexe: Code chaînage}
\begin{figure}[h]
    \begin{minipage}[c]{.46\linewidth}
        \centering
        \includegraphics[scale=0.45]{../../Downloads/TIPE (2)/image code/chainage2.jpg}

    \end{minipage}
\end{figure}
\end{frame}

\begin{frame}[fragile]
\frametitle{Annexe: Code chaînage}
\begin{figure}[h]
    \begin{minipage}[c]{.46\linewidth}
        \centering
        \includegraphics[scale=0.45]{../../Downloads/TIPE (2)/image code/chainage3.jpg}

    \end{minipage}
\end{figure}
\end{frame}

\begin{frame}[fragile]
\frametitle{Annexe: Code chaînage}
\begin{figure}[h]
    \begin{minipage}[c]{.46\linewidth}
        \centering
        \includegraphics[scale=0.45]{../../Downloads/TIPE (2)/image code/chainage4.jpg}

    \end{minipage}
\end{figure}
\end{frame}

\begin{frame}[fragile]
\frametitle{Annexe: Code chaînage}
\begin{figure}[h]
    \begin{minipage}[c]{.46\linewidth}
        \centering
        \includegraphics[scale=0.45]{../../Downloads/TIPE (2)/image code/chainage5.jpg}

    \end{minipage}
\end{figure}
\end{frame}

\begin{frame}[fragile]
\frametitle{Annexe: Code affiliation}
\begin{figure}[h]
    \begin{minipage}[c]{.46\linewidth}
        \centering
        \includegraphics[scale=0.45]{../../Downloads/TIPE (2)/image code/chainage6.jpg}

    \end{minipage}
\end{figure}
\end{frame}

\begin{frame}[fragile]
\frametitle{Annexe: Code affiliation}
\begin{figure}[h]
    \begin{minipage}[c]{.46\linewidth}
        \centering
        \includegraphics[scale=0.45]{../../Downloads/TIPE (2)/image code/chainage7.jpg}

    \end{minipage}
\end{figure}
\end{frame}

\begin{frame}[fragile]
\frametitle{Annexe: Code affiliation}
\begin{figure}[h]
    \begin{minipage}[c]{.46\linewidth}
        \centering
        \includegraphics[scale=0.45]{../../Downloads/TIPE (2)/image code/chainage8.jpg}

    \end{minipage}
\end{figure}
\end{frame}


\end{document}
