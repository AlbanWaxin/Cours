\documentclass[12pt]{report}
\usepackage[utf8]{inputenc}
\usepackage[T1]{fontenc}
\usepackage[fleqn]{amsmath}
\usepackage{amsfonts,amssymb,stmaryrd}
\usepackage[english]{babel}
\usepackage{pdfpages}

%=============Affichage=======================
\usepackage{fullpage}
\usepackage{mathtools}
\usepackage{lmodern}
\usepackage{xcolor}
\usepackage{enumitem}
\usepackage{tikz,tkz-tab}
\usepackage[ruled,vlined]{algorithm2e}
\usepackage{booktabs}
\usepackage{setspace}
\title{TD Statistiques 3A STI}
\author{Waxin Alban}

\definecolor{almond}{rgb}{0.94, 0.87, 0.8}
\definecolor{champagne}{rgb}{0.97, 0.91, 0.81}
\definecolor{dgreen}{rgb}{0.0, 0.5, 0.0}
\definecolor{bc}{rgb}{0.8588, 0.8980, 0.9450}

\setlength{\topmargin}{-1.5cm}
\setlength{\textheight}{25cm}
\setlength{\textwidth}{16cm}
\setlength{\oddsidemargin}{-1.5cm}
\setlength{\evensidemargin}{50cm}
\doublespacing  

\newcommand{\rd}[1]{\textcolor{red}{#1}}
\newcommand{\g}[1]{\textcolor{lime}{#1}}
\newcommand{\dg}[1]{\textcolor{dgreen}{#1}}
\newcommand{\blue}[1]{\textcolor{blue}{#1}}
\newcommand{\cy}[1]{\textcolor{cyan}{#1}}
\newcommand{\blz}{$\blacklozenge$}
\newcommand{\ns}{\\\indent\indent\vspace{0.25cm}}
\setcounter{secnumdepth}{5}% profondeur de la table des matières
\usepackage{titlesec}


\titleformat{\chapter}[frame]
{\Huge}
{\filright\rmfamily\bfseries\Huge\enspace\thechapter\enspace}
{18pt}
{\rmfamily\huge\bfseries\filcenter}
% rmfamily=roman, sffamily = sans serif ou ttfamily =type writer
\usepackage[many]{tcolorbox} % Creation de box collorable pour le texte non intégré
\newtcolorbox{mybox}{colback=bc,
colframe=black,arc=0mm,sharp corners= northwest,arc=10pt}

\newtcolorbox{demo}{colback=almond,
colframe=black,arc=0mm,sharp corners= northeast,arc=10pt}

\renewcommand*{\overrightarrow}[1]{\vbox{\halign{##\cr
 \tiny\rightarrowfill\cr\noalign{\nointerlineskip\vskip1pt}
 $#1\mskip2mu$\cr}}}

\newcommand{\rem}[1]
{
\subparagraph*{\underline{Remarque:#1}}\mbox{}\\
}

\newcommand{\props}[1]
{
\begin{mybox}
\textbf{\rd{\underline{\blz Propriété:} #1}}
\vspace{0.5cm}
\newline
}

\newcommand{\prope}
{
\end{mybox}
}

\newcommand{\scal}[2]
{
<#1|#2>
}

\newcommand{\defis}[1]
{
\begin{mybox}
\textbf{\rd{\underline{\blz Définition:} #1}}
\vspace{0.5cm}
\newline
}
\newcommand{\defie}
{
\end{mybox}
}
\newcommand{\demos}[1]
{
\begin{demo}
\textbf{\underline{\blz Démonstration:} #1}
\newline
}
\newcommand{\demoe}
{
\end{demo}
}
\newcommand{\exe}[1]
{
\subparagraph*{\underline{Exemple:#1}}\mbox{}\\
}

\newcommand{\vs}
{
\vspace{0.25cm}
}

\newcommand{\thms}[1]
{
\begin{mybox}
\textbf{\rd{\underline{\blz Théorème:} #1}}
\vspace{0.5cm}
\newline
}

\newcommand{\thme}
{
\end{mybox}
}

\newcommand{\coros}[1]
{
\begin{mybox}
\textbf{\rd{\underline{\blz Corolaire:} #1}}
\vspace{0.5cm}
\newline
}

\newcommand{\coroe}
{
\end{mybox}
}

\newcommand{\lems}[1]
{
\begin{mybox}
\textbf{\rd{\underline{\blz Lemme:} #1}}
\vspace{0.5cm}
\newline
}

\newcommand{\leme}
{
\end{mybox}
}
%=============================================

%\usepackage[cm]{aeguill}

%=============Mathématiques=================

%--------------Raccourcis:------------------
\newcommand{\R}{\mathbb{R}}
\newcommand{\C}{\mathbb{C}}
\newcommand{\N}{\mathbb{N}}
\newcommand{\Q}{\mathbb{Q}}
\newcommand{\Z}{\mathbb{Z}}
\newcommand{\K}{\mathbb{K}}
\newcommand{\M}{\mathcal{M}}
\newcommand{\nint}[1]{#1 \in \N}
\newcommand{\zint}[1]{#1 \in \N^*}
\newcommand{\limi}[1]{\underset{#1 \to \infty}{lim}}
\newcommand{\limn}[2]{\underset{#1 \to #2}{lim}}
\newcommand{\x}{\times}
\newcommand{\un}[1]{u_{#1}}
\newcommand{\uns}{(u_n)_\nint{n}}
\newcommand{\Sn}[1]{S_{#1}}
\newcommand{\Sns}{(S_n)_\nint{n}}
\newcommand{\ol}[1]{\overline{#1}}
\newcommand{\znz}{\Z/n\Z}
\renewcommand{\o}{\circ}

\newcommand{\seriegu}{\sum u_n}
\newcommand{\seriegv}{\sum v_n}
\newcommand{\harmonique}{\sum \frac{1}{n}}
\newcommand{\SRieman}{\sum \frac{1}{n^\alpha}}
\newcommand{\serie}[3]{\sum_{#1}^{#2}{#3}}
\newcommand{\satps}{série à terme positif}
\newcommand{\satp}{séries à termes positifs}
\newcommand{\pl}[1]{\mathbb{#1}[X]}
\newcommand{\som}[2]{\sum\limits_{#1}^{#2}}


\newcommand{\abs}[1]{\left\lvert#1\right\rvert}
\DeclarePairedDelimiter{\ceil}{\lceil}{\rceil}

%Format de fonctions:
\newcommand{\fct}[5]
	{
	  \begin{array}{ccccc}
		#1 & : & #2 & \to & #3 \\
	    && #4 & \mapsto & #5 \\
	  \end{array}
	}
\newcommand{\dfct}[2] {#1 \mapsto #2}

\renewcommand{\abs}[1]{|#1|}
\newcommand{\nm}[2]{ ||#1||_{#2} }
%===================TESTS===================

\begin{document}
\chapter{Rappels de probabilités}
\section{Exercice 1}
\section{Exercice 2}
\section{Exercice 3}
\section{Exercice 4}
\normalsize
\begin{enumerate}
    \item On tire un homme, donc \\
        $P_H(F) = \frac{P(H \cap F)}{P(H)} = \frac{\frac{Card(F\cap H)}{Card(\Omega)}}{\frac{Card(H)}{Card(\Omega)}}$ \\
        $= \frac{420}{700} = \frac{21}{35}$
    \item $P_F(H) = \frac{420}{495}$
    \item $P(H) = \frac{700}{1000}$
    \item $P(F) = \frac{495}{1000}$
    \item $P(H \cap F) = \frac{420}{1000}$
    \item $P(H \cap F) = P(H)P(F)$?\\
    $P(H\cap F) = \frac{420}{1000}$    |   $P(H)P(f)= \frac{7\times495}{10 \times 1000} \neq \frac{420}{1000}$\\
    Donc H et F non indépendants
    \item 
\end{enumerate}

\section{Exercice 5}

Rappel de tous les paramètres:\\
\begin{itemize}
    \item  A pièce venant du fournisseur A
    \item B pièce venant du fournisseur B
    \item D Défectueuse
\end{itemize}
$P(A) = 75 \% \Rightarrow P(B) = 1 - P(A) = 25\%$\\
$P_A(D) = 1 \% \Rightarrow P_A(\overline{D}) = 99 \%$\\
$P_B(D) = 2 \% \Rightarrow P_B(\overline{D}) = 98 \%$\\


\begin{enumerate}
    \item 
    \begin{align*}
        P_{\overline{D}}(A) &= \frac{P_{A}(\overline{D})P(A)}{P_A(\overline{D})P(A)+ P_{\overline{A}}(\overline{D})P(\overline{A})}\\
        &= \frac{99 \% 75 \%}{99 \% 75\% + 98 \% 25\%}\\
        &= 0.752    
    \end{align*}
    \item 
    \begin{align*}
        P_{D}(B) &=  \frac{P_{B}(D)P(B)}{P_B(D)P(B)+ P_{A}(D)P(A)}\\
        &= \frac{25 \% 2 \%}{25\% 2\% + 75\% 1\%} \\
        &= 0.4  
    \end{align*} 
    Donc $P_D(A) = 60 \% > P_D(B)$
\end{enumerate}

\section{}
\section{}
\section{}
\section{Exercice 9}
\begin{enumerate}
    \item 
    \begin{itemize}
        \item Loi de X:\\
        $X(\Omega) = \{ -1,1 \}$\\
        $P(X=-1) = P(X=-1, Y = -1) + P(X=-1, Y = 1)+ P(X=-1, Y=2) = \frac{3}{8}$\\
        $P(X=1) = \frac{1}{8}+ \frac{1}{8} +\frac{3}{8} = \frac{5}{8}$\\
        \item Loi de Y:\\
        $Y(\Omega) = \{ -1,1,2 \}$\\
        $P(Y = -1) = \frac{3}{8}$\\
        $P(Y = 1) = \frac{1}{8}$\\
        $P(Y = 2) = \frac{4}{8}$\\
        \item $E(X) = \sum_k k P(X=k) = -\frac{3}{8} + \frac{5}{8} = \frac{1}{4}$
        \item $E(Y) = \frac{3}{4}$
    \end{itemize}
    \item $cov(X,Y) = E(XY) - E(X)E(Y)$\\
    $XY(\Omega) = \{ -1,1,2,-2 \}$
    $E(XY) = \frac{1}{4} \times 1 + 0 \times -1 + \frac{1}{8} \times -2 + \frac{1}{8} \times (-1) + \frac{1}{8} + \frac{3}{8} \times 2 = \frac{3}{4}$\\
    $cov(X,Y) = \frac{3}{4} - \frac{3}{16} = \frac{9}{16} \neq 0$\\
    $cov(X,Y)$ non nul donc les variables ne sont pas indépendantes
    \item $P_{Y=-1}(X = -1) = \frac{P(X=-1,Y=-1)}{P(Y=-1)} = 2/3$\\    
    $P_{Y=-1}(X = 1) = \frac{P(X=1,Y=-1)}{P(Y=-1)} = 1/3$\\  
    \item \begin{align*}
        Var(X) &= E(X^2) - E(X)^2\\
        &= \sum k P(X^2=k)\\
        &= \sum k^2 P(X = k)\\
        &= \frac{3}{8} + \frac{5}{8} =1
    \end{align*}
\end{enumerate}

\section{ Exercice 13}
$f(x) = \begin{cases}
    5 (1-x)^4 , 0<x<1\\
    0,x>1
\end{cases}$\\
$Var(X) = E(X^2) - E(X)^2$\\
Soit X une variable aléatoire
X: demande d'essence en milliers de litres/semaine à cours d'essence ssi $X>V_0$\\
On veut 
\begin{align*}
    P(X> V_0) < 0,01 & \leftrightarrow \int_{V_0}^{\infty} f(x)dx < 0,01\\
    & \leftrightarrow \int_{V_0}^1 5(1-x)^4 dx < 0,01\\
    & \leftrightarrow [(x-1)^5]^1_{V_0} < 0,01\\
    & \leftrightarrow - (V_0 -1)^5 < 0,01\\
    & \leftrightarrow 1-V_0 < 0,01^{\frac{1}{5}}\\
    & \leftrightarrow  V_0 >  1- 0,01^{\frac{1}{5}}
\end{align*}

\chapter{Principales lois de probabilités}
\section{Exercice 1}
Soit X une variable aléatoire telle que $X \sim \mathcal{N}(160,30)$
\begin{enumerate}
    \item il ne s'agit pas en réalité de la durée de vie d'un tube mais plutot la moyenne d'un lot de tube.\\
    On le voit au fait que X suive une loi normale et non une loi exponentielle.
    \item 
        \begin{itemize}
            \item $P(X \leq  140)$:\\
            Si $X \sim \mathcal{N}(160;30)$ alors $\frac{X-160}{30} \sim \mathcal{N}(0;1)$\\
            \begin{align*}
                P(X \leq  140) &= P(\frac{X-160}{30} \leq \frac{140-160}{30})\\
                &= P(\frac{X-160}{30} \leq -0,67)
                & \text{Lecture dans la table $P(X \leq 0,67) = 0,7486$}\\
                &= P(\frac{X-160}{30} \geq 0,67)\\
                &= 1 - P(\frac{X-160}{30} < 0,67)\\
                &= 1- 0,7486\\
                &=0,2514\\
                & = 25,14%
            \end{align*}
            \item $P(X \geq 200)$:\\
            Si $X \sim \mathcal{N}(160;30)$ alors $\frac{X-160}{30} \sim \mathcal{N}(0;1)$\\
            \begin{align*}
                P(X \geq 200 ) &= P(\frac{X-160}{30} \geq \frac{200-160}{30})\\
                &= P(\frac{X-160}{30} \geq 1,33)
                & \text{Lecture dans la table $P(X < 0,133) = 0,9082$}\\
                &= P(\frac{X-160}{30} < 1,33)\\
                &= 1 - P(\frac{X-160}{30} < 1,33)\\
                &= 1- 0,9082\\
                &=0,0918\\
                & = 9,18%
            \end{align*}
            \item $P(130 \leq X \leq 190)$:\\
            Si $X \sim \mathcal{N}(160;30)$ alors $\frac{X-160}{30} \sim \mathcal{N}(0;1)$\\
            \begin{align*}
                P(X \leq 190) - P(X < 130) &= P(\frac{X-160}{30} \leq \frac{190-160}{30}) - P(\frac{X-160}{30} < \frac{130-160}{30}) \\
                &= P(\frac{X-160}{30} \leq 1) - P(\frac{X-160}{30} <-1)\\
                & \text{Lecture dans la table}
                &= P(\frac{X-160}{30} \leq 1) - (1- P(\frac{X-160}{30} \leq 1))\\
                &= 2 \times P(\frac{X-160}{30} \leq 1) -1\\
                &=0,6826\\
                & = 68,26%
            \end{align*}
        \end{itemize} 
    \item\begin{align*}
        P(X \leq a) =  0,9 &\leftrightarrow P(\frac{X-160}{30} \leq \frac{a-160}{30}) =0,9\\
        &\leftrightarrow  \frac{a-160}{30}  = 1,28
        &\leftrightarrow a = 198,4
    \end{align*}
    \item \begin{align*}
        P(X \geq b) = 0.8  &\leftrightarrow P(\frac{X-160}{30} \geq \frac{b-160}{30}) =0,8\\
        &\leftrightarrow 1 - P(\frac{X-160}{30} \leq \frac{b-160}{30}) = 0.8\\
        &\leftrightarrow P(\frac{X-160}{30} \leq \frac{b-160}{30}) = 0.2\\
        &\leftrightarrow 1 - P(\frac{X-160}{30} \leq -\frac{b-160}{30}) = 0.2\\
        &\leftrightarrow P(\frac{X-160}{30} \leq -\frac{b-160}{30}) = 0.8\\
        &\leftrightarrow  \frac{160-b}{30}  = 0,84\\
        &\leftrightarrow b = 134,8
    \end{align*}
    \item \begin{align*}
        P(X >200 | X>160) &= \frac{P(X>200) \cap P(X>160)}{P(X>160)}\\
        &= \frac{P(X > 200)}{P(X > 160)}\\
        &= 18,36%
    \end{align*}
\end{enumerate}

\section{Exercice 2}

$X \sim \mathcal{B}(400;0,1)$\\
$n = 400 \geq 30$\\
$np = 40 \geq 5$\\
$n(1-p) = 360 \geq 5$\\
$X \sim \mathcal{N}(40;\sqrt{40\times 0,9})$\\
$P(X\leq 30) = 0,0524 = 5,24\% avec B(400;0,1)$\\
\begin{align*}
    P(X \leq 30) &= P( \frac{X-40}{6} \leq \frac{30-40}{6})\\
    &= P(\frac{X-40}{6}\leq  -1,67)\\
    &= 1 - P(\frac{X-40}{6} \leq 1,67)\\
    &= 1-0,9525\\
    &=4,75\%   
\end{align*}

Correction de continuité

\begin{align*}
    P(x \leq 30) &= P(X=0) + P(X=1) + \dots + P(X =30)\\
    &= P( -0.5 \leq X \leq 0.5) + P( 0.5 \leq X \leq 1.5) + \dots + P( 29.5 \leq X \leq 30.5)\\
    &= P(-0.5 \leq X \leq 30.5)\\
    &\simeq P(X \leq 30.5)\\
    & = P(\frac{X-40}{6} \leq \frac{30,5-40}{6})\\
    & = P(\frac{X-40}{6} \leq -1,58)\\
    & = 1- P(\frac{X-40}{6} \leq 1,58)\\
    & = 5.71\%
\end{align*}
 
\chapter{Echantillonnage et Estimation}

\section{Exercice 1}

Pas le cas 1: car l'écart-type de la population n'est pas connu, on connait seulement l'écart-type de l'échantillon\\
Pas le cas 3: car l'échantillon est de taille supérieure à 30\\
Donc cas 2\\

L'intervalle de confianc est alors 
\begin{align*}
    I &= [\overline{x_e} - t_\alpha \frac{S_e}{\sqrt{n}}; \overline{x_e} + t_\alpha \frac{S_e}{\sqrt{n}}]\\
    &= [\overline{x_e} - t_\alpha \frac{\sigma_e}{\sqrt{n}}; \overline{x_e} + t_\alpha \frac{\sigma_e}{\sqrt{n-1}}]
\end{align*}

$\overline{x_e} = 3000$\\
$\sigma_e = 20$\\
$n = 49$\\
$1-\alpha \to t_\alpha = 1,645$\\
$I = [2995,25;3004,75]$

\section{Exercice 2}
La variance est connue $\sigma^2 = 0,12^2$ et la population est normale  donc c'est le cas 1\\
$I= [\overline{x_e} - t_\alpha \frac{\sigma}{\sqrt{n}}; \overline{x_e} + t_\alpha \frac{\sigma}{\sqrt{n}}]$\\
Longueur de $I = 2 t_\alpha \frac{\sigma}{\sqrt{n}}$\\
$\sigma = 0,12$\\
$t_\alpha = 2,576$ car $1-\alpha = 99 \%$\\
On veut 
\begin{align*}
    2 t_\alpha \frac{\sigma}{\sqrt{n}} \leq 6 10^2 &\Leftrightarrow \sqrt{n} \geq \frac{2 \times 2,576 \times 0,12}{0,06} = 10,3\\
    &\Leftrightarrow n \geq 106,09\\
    &\Leftrightarrow n = 107
\end{align*}

\section{Exercice 3}
Ici $p_e = \frac{15}{125}$ et $n = 125$\\
$np_e=15 \geq 5$\\
$n(1-p_e) = 110 \geq 5$\\
$I= [p_e - t_\alpha \frac{p_e(1-p_e)}{\sqrt{n}}; p_e + t_\alpha \frac{p_e(1-p_e)}{\sqrt{n}}]$\\
avec $\mathbb{N}(0;1) tableau p37 1-\alpha = 95 \%, t_\alpha = 1,96$\\
$I=[6,3\%,17,7\%]$\\

\section{Exercice 4}
On suppose que la population est normale, c'est à dire que le temps d'utilisation en minutes suit une loi normale\\
La variance est inconnue et l'échantillon est petit donc il s'agit d'un cas 3
\begin{align*}
    I &= [\overline{x_e} - t_\alpha \frac{S_e}{\sqrt{n}}; \overline{x_e} + t_\alpha \frac{S_e}{\sqrt{n}}]\\
    &= [\overline{x_e} - t_\alpha \frac{\sigma_e}{\sqrt{n}}; \overline{x_e} + t_\alpha \frac{\sigma_e}{\sqrt{n-1}}]
\end{align*}

$\overline{x_e},\sigma_e$ et $t_{\alpha,n-1}$ à calculer  
loi de Student à partir de $1-\alpha = 99 \%$\\
$\overline{x_e} =26,8$\\
\begin{align*}
    \sigma_e^2 &= \overline{x_e^2} - \overline{x_e}^2\\
    &= 724,23 - 718,24\\
    &=6\\
\end{align*}

$t_{\alpha,n-1} = t_{\alpha,8} = 3,355$ par lecture de la table de Student\\
$I = [23,9;29,7]$

\section{Exercice 5}

Il y a 400 mesures donc n>30, on peut donc supposer que $\overline{X} \sim \mathcal{N}$\\
L'écart type de l'échantillon est inconnue\\
L'échantillon est assez grand il s'agit donc d'un cas 2\\
Donc $\frac{(\overline{X}-m)\sqrt{n}}{\hat{S}} \sim \mathcal{N}(0,1)$\\
$\overline{X} \sim \mathcal{N}(m,\frac{s_e}{\sqrt{n}})$\\
Donc approximativement $\overline{X} \sim \mathcal{N}(64;\frac{8}{\sqrt{400}}) = \mathcal{N}(64;0,4)$\\

\begin{itemize}
    \item $1- \alpha = 50\%$ \\
    $I=[\overline{x_e} - t_\alpha \frac{s_e}{\sqrt{n}};\overline{x_e} + t_\alpha \frac{s_e}{\sqrt{n}}]= [63.73;64.27]$
    \item $1-\alpha = 90\% \Rightarrow t_\alpha = 1,645$\\
    $I=[\overline{x_e} - t_\alpha \frac{s_e}{\sqrt{n}};\overline{x_e} + t_\alpha \frac{s_e}{\sqrt{n}}] = [63.34;64.66]$
    \item $1-\alpha = 99\% \Rightarrow t_\alpha = 1,2576$\\
    $I=[\overline{x_e} - t_\alpha \frac{s_e}{\sqrt{n}};\overline{x_e} + t_\alpha \frac{s_e}{\sqrt{n}}]= [62.97;65,03]$
\end{itemize}

Déterminer $t_\alpha$:\\
\begin{align*}
    P(- t_\alpha \leq Y \leq t_\alpha) = 1-\alpha &\Leftrightarrow P(Y \leq t_\alpha) - P(Y < -t_\alpha) = 1-\alpha\\
    &\Leftrightarrow P(Y \leq t_\alpha) - (1-P(Y \leq t_\alpha)) = 1 - \alpha\\
    &\Leftrightarrow 2P(Y \leq t_\alpha) -1 = 1- \alpha\\
    &\Leftrightarrow P(Y \leq t_\alpha)  = 0,75\\
    &\text{Par lecture de la table} t_\alpha = 0.67 
\end{align*}

\section{Exercice 6}

Soit $\fct{g}{\R}{\R}{p}{\prod_{i=1}^n (1-p)^{i-1}p}$\\
D'ou 
\begin{align*}
    ln(g)(p) = \sum_{i=1}^n ln((1-p)^{x_i-1}p)\\
    ln(g)(p) = \sum_{i=1}^n ln((1-p)^{x_i-1}) + \sum_{i=1}^n ln(p)\\
    ln(g)(p) = \sum_{i=1}^{n} ln((1-p)^{x_i-1}) + nln(p)\\
    ln(g)(p) = ln(1-p)(-n + \sum_{i=1}^{n} x_i)  + nln(p)\\
\end{align*}
Dérivons:
\begin{align*}
    (ln(g))'(p) = \frac{n}{p} - \frac{1}{1-p}\sum_{i=1}^{n} x_i + \frac{n}{1-p}\\
    (ln(g))'(p) = 0 &\Leftrightarrow \frac{n}{p} - \frac{1}{1-p}\sum_{i=1}^{n} x_i + \frac{n}{1-p} =0\\
    \Leftrightarrow n(1-p) = p (\sum_{i=1}^{n} x_i  - n)\\
    \Leftrightarrow n = p \sum_{i=1}^n x_i\\
    \Leftrightarrow p = \frac{n}{\sum_{i=1}^n x_i}
\end{align*}
ReDérivons pour savoir si c'est un maximum ou un minimum:\\
\begin{align*}
    (ln(g))''(p) = \frac{-n}{p^2} - \frac{1}{(p-1)^2}\sum_{i=1}^n x_i  + \frac{n}{(1-p)^2}\\
    (ln(g))''(p) = \frac{-n}{p^2} - \frac{1}{(p-1)^2}(\sum_{i=1}^n x_i   -n)\\
    \leq 0
\end{align*}

\chapter{Tests paramétriques}

\section{Exercice 1}
\section{Exercice 2}

\begin{enumerate}
    \item $H_0 , \mu = 100$ \\
          $H_1 , \mu \neq 100$ 
    \item $\alpha = 5 \%$
    \item Variance inconnue + petit échantillon + population normale $\Rightarrow$ cas 3
    \item $\frac{}{}$
\end{enumerate}

\section{Exercice 3}

\begin{enumerate}
    \item $H_0 , \mu = 925$ \\
          $H_1 , \mu > 925$ 

\end{enumerate}

\begin{enumerate}
    \item s
    \item s
    \item s
    \item fromage
    \item  $J = [(\mu_0 - \frac{t_\alpha \sigma}{\sqrt{n}};\mu_0 + \frac{t_\alpha \sigma}{\sqrt{n}})]$
    \item On accepte $H_0 \Leftrightarrow \overline{x_e}$
\end{enumerate}

\section{Exercice 8}

Test bilatéral moyenne:
$\begin{cases}
    H_0, \mu = 70\\
    H_1, \mu \neq 70
\end{cases}$

Variance connue ($\theta = 3$) et population normale donc cas 1\\
Soit $J = [70-t_\alpha \frac{3}{\sqrt{25}},70+t_\alpha \frac{3}{\sqrt{25}}]$\\
$\overline{x_e} = 69$\\
On résout $69 = 70 - 3 \frac{t_{\alpha_p} }{\sqrt{25}} \Leftrightarrow t_{\alpha_p} = \frac{5}{3}*1 = 1,67$

Rappel:\\
Si Y suit $\mathcal{N}(0,1), P(-t_{\alpha} \leq y \leq t_{\alpha}) = 1-\alpha$

Calcul de alpha

$P(-t_{\alpha} \leq y \leq t_{\alpha}) = 1-\alpha \Leftrightarrow 2 P(Y \leq t_{\alpha}) -1 = 1-\alpha$

D'ou: $\alpha_p =2 - 2P(Y \leq t_\alpha) = 9.5$\\
Donc si $\alpha = 5\% : \overline{x_e} \in J$ donc on accepte $H_0$\\
si $\alpha > \alpha_p$ on accepte pas $H_0$ sinon oui

\chapter*{Tests d'ajustement}

\section{Exercice 1}

X le nombre de bonnes réponses so un observateur répond au hasard\\
X suit $\mathcal{B}(6,1/4)$\\
X peut prendre les valeurs 1,2,3,4,5,6\\
\begin{tabular}{|c|c|c|c|c|c|c|c|}
    \hline
    &0&1&2&3&4&5&6\\
    \hline
    $N_k$&6&20&14&8&2&0&0\\
    \hline  
    $np_k$&8.9&17.8&14.83&6.59&1.65&0.23&\\
    \hline  
\end{tabular}
$np_k = 50 * P(X=k)$: effectifs théoriques\\
\begin{tabular}{|c|c|c|c|c|}
    \hline
    &0&1&2&3 et +\\
    \hline
    $N_k$&6&20&14&10\\
    \hline  
    $np_k$&8.9&17.8&14.83&8.47\\
    \hline  
    &0.945&0.272&0.043&0.265\\
    \hline
\end{tabular}
et $D^2 = \sum_k \frac{(N_k-np_k)^2}{np_k} = 1.525$ 

Si $D^2 < \chi_{\alpha,\nu}^2$, on accepte $H_0$: ils ont répondu au hasard cad dire X suit $\mathcal{B}(6,1/4)$\\
Pour $\alpha = 5\%$ et $\nu = 3$, $\chi_{\alpha,\nu}^2 = 7.8147$\\
Or $1.525 < 7.8147$ donc on accepte $H_0$

\section{Exercice 2}

\begin{tabular}{|c|c|c|c|c|c|c|c|c|c|c|}
    \hline
    $N_k$&15&10&14&8&11&11&11&13&13&14\\
    \hline  
    $np_k$&12&12&12&12&12&12&12&12&12&12\\
    \hline  
    &0.75&0.33&0.33&1.33&0.083&0.083&0.083&0.083&0.83&0.33\\
    \hline
\end{tabular}
Et $D^2 = 3.5$\\
$N = 120 \geq 5$\\
nb classes 10 > 4
$12 = np_k > 5$
Pour $\alpha = 5\%$ et $\nu = 9$, $\chi_{\alpha,\nu}^2 = 16.919$\\
$D^2 < \chi_{\alpha,\nu}^2$ on valide le programme informatique

\chapter*{Simulations numériques de lois}

\section{Exercice 1}

$P(\lfloor Y \rfloor = k) = P(k \leq Y < k+1)= \int_{k}^{k+1} \lambda e^{-\lambda x} dx = e^{-\lambda k} (1-e^{-\lambda})$) 
$P(N = k) = P(\lfloor Y \rfloor = k-1)= e^{-\lambda (k-1)} (1-e^{-\lambda})$\\
On reconnait $\mathcal{G}(1-e^{-\lambda})$\\
Si X suit $U([0,1])$ alors (cf poly) $\frac{-1}{\lambda}ln(x)$ sit Exp($\lambda$)

$p = 1-e^{-\lambda} \Leftrightarrow \lambda = -ln(1-p)$
Avec $\lambda = ln(1-p)$ : $1+ \lfloor \frac{ln(x)}{ln(1-p)} \rfloor$ suit $\mathcal{G}(p)$

\section{Exercice 2}

$N(t)$ suit $\mathcal{P}(\lambda t)$\\

$P(N(20) = 2 | N(60) = 2) =\Large \frac{P(N(60)=2 | N(20) = 2)P(N(20)=2)}{P(N(60)=2)} = \frac{P(N(40)=0)P(N(20)=2)}{P(N(60)=2)}\normalsize = 1/9$\\
$P(N(20)\geq | N(60) = 2) = 1 - P(N(20)=0 | N(60)=2) = 1 - \frac{P(N(60)=2 | N(20) = 0)P(N(20)=0)}{P(N(60)=2)}$\\
$=\frac{P(N(40)=2)P(N(20)=0)}{P(N(60)=2)} = $

\section{Exercice 3}

D'après exercice 1\\
$1 + \lfloor \frac{ln(x)}{ln(1-p)} \rfloor$ suit $\mathcal{G}(p)$ si $X$ suit $\mathcal{U}([0,1])$ \\
en particulier avec $p =1/2$
$1 + \lfloor \frac{ln(x)}{ln(1-p)} \rfloor$ suit $\mathcal{G}(1/2)$\\
\begin{tabular}{|c|c|c|c|c|c|c|c|}
    \hline
    $k$&1&2&3&4&5&6&7\\
    \hline  
    $N_k$&22&12&6&8&1&0&1\\
    \hline
\end{tabular}

$H_0 : Y$ suit $\mathcal{G}(1/2)$\\
$Y$ à valeur dans $\N^*$\\
$P(Y=1) = 1/2$,$P(Y=2) = 1/4$,$P(Y=3)=1/8$\\
\begin{tabular}{|c|c|c|c|c|c|c|c|}
    \hline
    $k$&1&2&3&4&5&6&7\\
    \hline  
    $N_k$&22&12&6&8&1&0&1\\
    \hline
    $np_k$&25&12.5&6.25&6.25&&&\\
    \hline
\end{tabular}\\

$\chi_{\alpha,\mu}^2 = 7,8147$\\
Donc $D^2 = \frac{(22-25)^2}{25} + \frac{(12-12,5)^2}{12.5} + \frac{(6-6,25)^2}{6,25} + \frac{10-6,25}{6,25} = 2,64$

\chapter*{Annexe}

\begin{tabular}{|l|c|c|c|c|c|c|c|c|c|c|}
    \hline
    x & 0,00 & 0,01 & 0,02 & 0,03 & 0,04 & 0,05 & 0,06 & 0,07 & 0,08 & 0,09\\ 
    \hline 
0,0 & 0.5000 & 0.5040 & 0.5080 & 0.5120 & 0.5160 & 0.5199 & 0.5239& 0.5279& 0.5319& 0.5359\\
\hline
0.1 & 0.5398 & 0.5438 & 0.5478 & 0.5517 & 0.5557 & 0.5596 & 0.5636& 0.5675& 0.5714& 0.5753\\
\hline
0.2 & 0.5793 & 0.5832 & 0.5871 & 0.5910 & 0.5948 & 0.5987 & 0.6026& 0.6064& 0.6103& 0.6141\\
\hline
0.3 & 0.6179 & 0.6217 & 0.6255 & 0.6293 & 0.6331 & 0.6368 & 0.6406& 0.6443& 0.6480& 0.6517\\
\hline
0.4 & 0.6554 & 0.6591 & 0.6628 & 0.6664 & 0.6700 & 0.6736 & 0.6772& 0.6808& 0.6844& 0.6879\\
\hline
0.5 & 0.6915 & 0.6950 & 0.6985 & 0.7019 & 0.7054 & 0.7088 & 0.7123& 0.7157& 0.7190& 0.7224\\
\hline
0.6 & 0.7257 & 0.7291 & 0.7324 & 0.7357 & 0.7389 & 0.7422 & 0.7454& 0.7486& 0.7517& 0.7549\\
\hline
0.7 & 0.7580 & 0.7611 & 0.7642 & 0.7673 & 0.7704 & 0.7734 & 0.7764& 0.7794& 0.7823& 0.7852\\
\hline
0.8 & 0.7881 & 0.7910 & 0.7939 & 0.7967 & 0.7995 & 0.8023 & 0.8051& 0.8078& 0.8106& 0.8133\\
\hline
0.9 & 0.8159 & 0.8186 & 0.8212 & 0.8238 & 0.8264 & 0.8289 & 0.8315& 0.8340& 0.8365& 0.8389\\
\hline
1.0 & 0.8413 & 0.8438 & 0.8461 & 0.8485 & 0.8508 & 0.8531 & 0.8554& 0.8577& 0.8599& 0.8621\\
\hline!
1.1 & 0.8643 & 0.8665 & 0.8686 & 0.8708 & 0.8729 & 0.8749 & 0.8770& 0.8790& 0.8810& 0.8830\\
\hline
1.2 & 0.8849 & 0.8869 & 0.8888 & 0.8907 & 0.8925 & 0.8944 & 0.8962& 0.8980& 0.8997& 0.9015\\
\hline
1.3 & 0.9032 & 0.9049 & 0.9066 & 0.9082 & 0.9099 & 0.9115 & 0.9131& 0.9147& 0.9162& 0.9177\\
\hline
1.4 & 0.9192 & 0.9207 & 0.9222 & 0.9236 & 0.9251 & 0.9265 & 0.9279& 0.9292& 0.9306& 0.9319\\
\hline
1.5 & 0.9332 & 0.9345 & 0.9357 & 0.9370 & 0.9382 & 0.9394 & 0.9406& 0.9418& 0.9429& 0.9441\\
\hline
1.6 & 0.9452 & 0.9463 & 0.9474 & 0.9484 & 0.9495 & 0.9505 & 0.9515& 0.9525& 0.9535& 0.9545\\
\hline
1.7 & 0.9554 & 0.9564 & 0.9573 & 0.9582 & 0.9591 & 0.9599 & 0.9608& 0.9616& 0.9625& 0.9633\\
\hline
1.8 & 0.9641 & 0.9649 & 0.9656 & 0.9664 & 0.9671 & 0.9678 & 0.9686& 0.9693& 0.9699& 0.9706\\
\hline
1.9 & 0.9713 & 0.9719 & 0.9726 & 0.9732 & 0.9738 & 0.9744 & 0.9750& 0.9756& 0.9761& 0.9767\\
\hline
2.0 & 0.9772 & 0.9778 & 0.9783 & 0.9788 & 0.9793 & 0.9798 & 0.9803& 0.9808& 0.9812& 0.9817\\
\hline
2.1 & 0.9821 & 0.9826 & 0.9830 & 0.9834 & 0.9838 & 0.9842 & 0.9846& 0.9850& 0.9854& 0.9857\\
\hline
2.2 & 0.9861 & 0.9864 & 0.9868 & 0.9871 & 0.9875 & 0.9878 & 0.9881& 0.9884& 0.9887& 0.9890\\
\hline
2.3 & 0.9893 & 0.9896 & 0.9898 & 0.9901 & 0.9904 & 0.9906 & 0.9909& 0.9911& 0.9913& 0.9916\\
\hline
2.4 & 0.9918 & 0.9920 & 0.9922 & 0.9925 & 0.9927 & 0.9929 & 0.9931& 0.9932& 0.9934& 0.9936\\
\hline
2.5 & 0.9938 & 0.9940 & 0.9941 & 0.9943 & 0.9945 & 0.9946 & 0.9948& 0.9949& 0.9951& 0.9952\\
\hline
2.6 & 0.9953 & 0.9955 & 0.9956 & 0.9957 & 0.9959 & 0.9960 & 0.9961& 0.9962& 0.9963& 0.9964\\
\hline
2.7 & 0.9965 & 0.9966 & 0.9967 & 0.9968 & 0.9969 & 0.9970 & 0.9971& 0.9972& 0.9973& 0.9974\\
\hline
2.8 & 0.9974 & 0.9975 & 0.9976 & 0.9977 & 0.9977 & 0.9978 & 0.9979& 0.9979& 0.9980& 0.9981\\
\hline
2.9  &0.9981  &0.9982  &0.9982  &0.9983  &0.9984  &0.9984  &0.9985 &0.9985 &0.9986 &0.9986 \\
\hline
\end{tabular}
\end{document}