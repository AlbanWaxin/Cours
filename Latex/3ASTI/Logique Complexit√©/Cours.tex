\documentclass[12pt]{report}
\usepackage[utf8]{inputenc}
\usepackage[T1]{fontenc}
\usepackage[fleqn]{amsmath}
\usepackage{amsfonts,amssymb,stmaryrd}
\usepackage[english]{babel}
\usepackage{pdfpages}

%=============Affichage=======================
\usepackage{fullpage}
\usepackage{mathtools}
\usepackage{lmodern}
\usepackage{xcolor}
\usepackage{enumitem}
\usepackage{tikz,tkz-tab}
\title{Cours Mathématiques MP*}
\author{Waxin Alban}

\definecolor{almond}{rgb}{0.94, 0.87, 0.8}
\definecolor{champagne}{rgb}{0.97, 0.91, 0.81}
\definecolor{dgreen}{rgb}{0.0, 0.5, 0.0}
\definecolor{bc}{rgb}{0.8588, 0.8980, 0.9450}

\setlength{\topmargin}{-1.5cm}
\setlength{\textheight}{25cm}
\setlength{\textwidth}{16cm}
\setlength{\oddsidemargin}{-1.5cm}
\setlength{\evensidemargin}{50cm}

\newcommand{\rd}[1]{\textcolor{red}{#1}}
\newcommand{\g}[1]{\textcolor{lime}{#1}}
\newcommand{\dg}[1]{\textcolor{dgreen}{#1}}
\newcommand{\blue}[1]{\textcolor{blue}{#1}}
\newcommand{\cy}[1]{\textcolor{cyan}{#1}}
\newcommand{\blz}{$\blacklozenge$}
\newcommand{\ns}{\\\indent\indent\vspace{0.25cm}}
\setcounter{secnumdepth}{5}% profondeur de la table des matières
\usepackage{titlesec}


\titleformat{\chapter}[frame]
{\Huge}
{\filright\rmfamily\bfseries\Huge\enspace\thechapter\enspace}
{18pt}
{\rmfamily\huge\bfseries\filcenter}
% rmfamily=roman, sffamily = sans serif ou ttfamily =type writer
\usepackage[many]{tcolorbox} % Creation de box collorable pour le texte non intégré
\newtcolorbox{mybox}{colback=bc,
colframe=black,arc=0mm,sharp corners= northwest,arc=10pt}

\newtcolorbox{demo}{colback=almond,
colframe=black,arc=0mm,sharp corners= northeast,arc=10pt}

\renewcommand*{\overrightarrow}[1]{\vbox{\halign{##\cr
 \tiny\rightarrowfill\cr\noalign{\nointerlineskip\vskip1pt}
 $#1\mskip2mu$\cr}}}

\newcommand{\rem}[1]
{
\subparagraph*{\underline{Remarque:#1}}\mbox{}\\
}

\newcommand{\props}[1]
{
\begin{mybox}
\textbf{\rd{\underline{\blz Propriété:} #1}}
\vspace{0.5cm}
\newline
}

\newcommand{\prope}
{
\end{mybox}
}

\newcommand{\scal}[2]
{
<#1|#2>
}

\newcommand{\defis}[1]
{
\begin{mybox}
\textbf{\rd{\underline{\blz Définition:} #1}}
\vspace{0.5cm}
\newline
}
\newcommand{\defie}
{
\end{mybox}
}
\newcommand{\demos}[1]
{
\begin{demo}
\textbf{\underline{\blz Démonstration:} #1}
\newline
}
\newcommand{\demoe}
{
\end{demo}
}
\newcommand{\exe}[1]
{
\subparagraph*{\underline{Exemple:#1}}\mbox{}\\
}

\newcommand{\vs}
{
\vspace{0.25cm}
}

\newcommand{\thms}[1]
{
\begin{mybox}
\textbf{\rd{\underline{\blz Théorème:} #1}}
\vspace{0.5cm}
\newline
}

\newcommand{\thme}
{
\end{mybox}
}

\newcommand{\coros}[1]
{
\begin{mybox}
\textbf{\rd{\underline{\blz Corolaire:} #1}}
\vspace{0.5cm}
\newline
}

\newcommand{\coroe}
{
\end{mybox}
}

\newcommand{\lems}[1]
{
\begin{mybox}
\textbf{\rd{\underline{\blz Lemme:} #1}}
\vspace{0.5cm}
\newline
}

\newcommand{\leme}
{
\end{mybox}
}
%=============================================

%\usepackage[cm]{aeguill}

%=============Mathématiques=================

%--------------Raccourcis:------------------
\newcommand{\R}{\mathbb{R}}
\newcommand{\C}{\mathbb{C}}
\newcommand{\N}{\mathbb{N}}
\newcommand{\Q}{\mathbb{Q}}
\newcommand{\Z}{\mathbb{Z}}
\newcommand{\K}{\mathbb{K}}
\newcommand{\M}{\mathcal{M}}
\newcommand{\nint}[1]{#1 \in \N}
\newcommand{\zint}[1]{#1 \in \N^*}
\newcommand{\limi}[1]{\underset{#1 \to \infty}{lim}}
\newcommand{\limn}[2]{\underset{#1 \to #2}{lim}}
\newcommand{\x}{\times}
\newcommand{\un}[1]{u_{#1}}
\newcommand{\uns}{(u_n)_\nint{n}}
\newcommand{\Sn}[1]{S_{#1}}
\newcommand{\Sns}{(S_n)_\nint{n}}
\newcommand{\ol}[1]{\overline{#1}}
\newcommand{\znz}{\Z/n\Z}
\renewcommand{\o}{\circ}

\newcommand{\seriegu}{\sum u_n}
\newcommand{\seriegv}{\sum v_n}
\newcommand{\harmonique}{\sum \frac{1}{n}}
\newcommand{\SRieman}{\sum \frac{1}{n^\alpha}}
\newcommand{\serie}[3]{\sum_{#1}^{#2}{#3}}
\newcommand{\satps}{série à terme positif}
\newcommand{\satp}{séries à termes positifs}
\newcommand{\pl}[1]{\mathbb{#1}[X]}
\newcommand{\som}[2]{\sum\limits_{#1}^{#2}}


\newcommand{\abs}[1]{\left\lvert#1\right\rvert}
\DeclarePairedDelimiter{\ceil}{\lceil}{\rceil}

%Format de fonctions:
\newcommand{\fct}[5]
	{
	  \begin{array}{ccccc}
		#1 & : & #2 & \to & #3 \\
	    && #4 & \mapsto & #5 \\
	  \end{array}
	}
\newcommand{\dfct}[2] {#1 \mapsto #2}

\renewcommand{\abs}[1]{|#1|}
\newcommand{\nm}[2]{ ||#1||_{#2} }
%===================TESTS===================

\begin{document}

\chapter{Logiques propositionnelles}
\chapter{ Notions générales}
\chapter{Logique du Premier Ordre}
\section{Définitions}

\defis{}
x - terme: nomme un objet\\
P - prédicat : exprime 
\defie

\underline{Exercice}
\begin{enumerate}
\item Marcus a vu un éléphant:\\
		Termes:
		\begin{itemize}
			\item $m$ : Marcus
			
		\end{itemize}
		Prédicats:\\
		\begin{itemize}
			\item $E(x)$ : $x$ est un éléphant
			\item $Voir(x,y)$ : $x$ a vu $y$
		\end{itemize}
		Réponse:\\
		$\exists x, (E(x) \wedge Voir(m,x))$
\item Annika s'est endormie:\\
		Termes:
		\begin{itemize}
			\item $a$ : Annika
		\end{itemize}
		Prédicats:\\
		\begin{itemize}
			\item $D(x)$ : $x$ s'est endormie
		\end{itemize}
		Réponse:\\
		$D(a)$
\newpage
\item Quelqu'un m'a frappé\\
		Termes:
		\begin{itemize}
			\item $m$ : Moi
		\end{itemize}
		Prédicats:\\
		\begin{itemize}
			\item $F(x,y)$ : $x$ à frapper $y$
			\item (pas obligé depend du contexte), $H(x) : x$ est un humain
		\end{itemize}
		Réponse:\\
		$\exists x (F(x,m))$
\item Personne n'est venu à ma fête:\\
		Termes:
		\begin{itemize}
			\item $m$: Moi
		\end{itemize}
		Prédicats:\\
		\begin{itemize}
			\item $F(x,y)$ : $x$ n'est pas venu a la fete de y
			\item (pas obligé depend du contexte), $H(x) : x$ est un humain
			\item (version sans negation), $F2(x,y)$ : $x$ n'est pas venu a la fete de $y$
		\end{itemize}
		Réponse:\\
		$\forall x, F(x,m)$\\
		$\forall x, (H(x) \Rightarrow F(x,y)$\\
		$\forall x, ( H(x) \Rightarrow \neg F2(x,y)$\\
\item Marcus m'a donné un livre, donc quelqu un m'a donné un livre:\\
		Termes:
		\begin{itemize}
			\item $m$: Marcus
			\item $h$: Moi
		\end{itemize}
		Prédicats:\\
		\begin{itemize}
			\item $D(x,y)$ : $x$  à  donner un livre à $y$ 
		\end{itemize}
		Réponse:\\
		$D(m,h) \Rightarrow (\exists x  D(x,h))$ 		
\item tout le monde aime le chocolat:\\
		Termes:
		Prédicats:\\
		\begin{itemize}
			\item $A(x)$ : $x$ aime le chocolat 
		\end{itemize}
		Réponse:\\
		$\forall x, A(x)$ 		
\end{enumerate}

\subsection{Termes,Fonctions,Prédicat,Atome}

\defis{}
\begin{itemize}
\item Termes
	\begin{itemize}
		\item Les variables: u,v,w,x,y,z commence par une minuscule
		\item les symboles de fonctions: commencent par une minuscule 
		\item clos s'il n'y a pas de nom de variables
	\end{itemize}
\item Fonctions: Symbole de fonction + nombre de termes correspondants $f(x)$ (nombre de terme = arrité)
\item Prédicat: Symbole de prédicat (Majuscule) + nombre de termes correspondants (nombre de terme = arrité) 
\item Atome: Prédicat ou Du type (s = t),
\end{itemize}
\defie

\props{Création de formules}
Prédicats,arrité: (R,0), (P,1),(Q,2), (=,2)
A := R| P(t)|
\prope

Exercice:

Définir le language pour l'énoncé: Socrate est un homme. Tous les hommes sont mortels. Donc socrate est mortel

$\mathcal{F} =  \lbrace (s,0)\rbrace$\\
$\mathcal{P} = \lbrace (H,1),(M,1)\rbrace$\\
Termes: $t := s | x$\\
Formules: $A := H(t) |M(t)| t=t| \bot | \top | \neg A| A \wedge A | A \vee A | A \to A| \forall x, A| \exists x, A$\\
Réponse: $(H(s) \wedge \forall x (H(x) \to M(x))) \to M(s)$  

\subsection{Vocabulaire}
\subsection{Sémantique}

Exercice:
\end{document}
