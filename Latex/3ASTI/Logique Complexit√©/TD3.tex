\documentclass[12pt]{report}
\usepackage[utf8]{inputenc}
\usepackage[T1]{fontenc}
\usepackage[fleqn]{amsmath}
\usepackage{amsfonts,amssymb,stmaryrd}
\usepackage[english]{babel}
\usepackage{pdfpages}

%=============Affichage=======================
\usepackage{fullpage}
\usepackage{mathtools}
\usepackage{lmodern}
\usepackage{xcolor}
\usepackage{enumitem}
\usepackage{tikz,tkz-tab}
\title{Cours Mathématiques MP*}
\author{Waxin Alban}

\definecolor{almond}{rgb}{0.94, 0.87, 0.8}
\definecolor{champagne}{rgb}{0.97, 0.91, 0.81}
\definecolor{dgreen}{rgb}{0.0, 0.5, 0.0}
\definecolor{bc}{rgb}{0.8588, 0.8980, 0.9450}

\setlength{\topmargin}{-1.5cm}
\setlength{\textheight}{25cm}
\setlength{\textwidth}{16cm}
\setlength{\oddsidemargin}{-1.5cm}
\setlength{\evensidemargin}{50cm}

\newcommand{\rd}[1]{\textcolor{red}{#1}}
\newcommand{\g}[1]{\textcolor{lime}{#1}}
\newcommand{\dg}[1]{\textcolor{dgreen}{#1}}
\newcommand{\blue}[1]{\textcolor{blue}{#1}}
\newcommand{\cy}[1]{\textcolor{cyan}{#1}}
\newcommand{\blz}{$\blacklozenge$}
\newcommand{\ns}{\\\indent\indent\vspace{0.25cm}}
\setcounter{secnumdepth}{5}% profondeur de la table des matières
\usepackage{titlesec}


\titleformat{\chapter}[frame]
{\Huge}
{\filright\rmfamily\bfseries\Huge\enspace\thechapter\enspace}
{18pt}
{\rmfamily\huge\bfseries\filcenter}
% rmfamily=roman, sffamily = sans serif ou ttfamily =type writer
\usepackage[many]{tcolorbox} % Creation de box collorable pour le texte non intégré
\newtcolorbox{mybox}{colback=bc,
colframe=black,arc=0mm,sharp corners= northwest,arc=10pt}

\newtcolorbox{demo}{colback=almond,
colframe=black,arc=0mm,sharp corners= northeast,arc=10pt}

\renewcommand*{\overrightarrow}[1]{\vbox{\halign{##\cr
 \tiny\rightarrowfill\cr\noalign{\nointerlineskip\vskip1pt}
 $#1\mskip2mu$\cr}}}

\newcommand{\rem}[1]
{
\subparagraph*{\underline{Remarque:#1}}\mbox{}\\
}

\newcommand{\props}[1]
{
\begin{mybox}
\textbf{\rd{\underline{\blz Propriété:} #1}}
\vspace{0.5cm}
\newline
}

\newcommand{\prope}
{
\end{mybox}
}

\newcommand{\scal}[2]
{
<#1|#2>
}

\newcommand{\defis}[1]
{
\begin{mybox}
\textbf{\rd{\underline{\blz Définition:} #1}}
\vspace{0.5cm}
\newline
}
\newcommand{\defie}
{
\end{mybox}
}
\newcommand{\demos}[1]
{
\begin{demo}
\textbf{\underline{\blz Démonstration:} #1}
\newline
}
\newcommand{\demoe}
{
\end{demo}
}
\newcommand{\exe}[1]
{
\subparagraph*{\underline{Exemple:#1}}\mbox{}\\
}

\newcommand{\vs}
{
\vspace{0.25cm}
}

\newcommand{\thms}[1]
{
\begin{mybox}
\textbf{\rd{\underline{\blz Théorème:} #1}}
\vspace{0.5cm}
\newline
}

\newcommand{\thme}
{
\end{mybox}
}

\newcommand{\coros}[1]
{
\begin{mybox}
\textbf{\rd{\underline{\blz Corolaire:} #1}}
\vspace{0.5cm}
\newline
}

\newcommand{\coroe}
{
\end{mybox}
}

\newcommand{\lems}[1]
{
\begin{mybox}
\textbf{\rd{\underline{\blz Lemme:} #1}}
\vspace{0.5cm}
\newline
}

\newcommand{\leme}
{
\end{mybox}
}
%=============================================

%\usepackage[cm]{aeguill}

%=============Mathématiques=================

%--------------Raccourcis:------------------
\newcommand{\R}{\mathbb{R}}
\newcommand{\C}{\mathbb{C}}
\newcommand{\N}{\mathbb{N}}
\newcommand{\Q}{\mathbb{Q}}
\newcommand{\Z}{\mathbb{Z}}
\newcommand{\K}{\mathbb{K}}
\newcommand{\M}{\mathcal{M}}
\newcommand{\nint}[1]{#1 \in \N}
\newcommand{\zint}[1]{#1 \in \N^*}
\newcommand{\limi}[1]{\underset{#1 \to \infty}{lim}}
\newcommand{\limn}[2]{\underset{#1 \to #2}{lim}}
\newcommand{\x}{\times}
\newcommand{\un}[1]{u_{#1}}
\newcommand{\uns}{(u_n)_\nint{n}}
\newcommand{\Sn}[1]{S_{#1}}
\newcommand{\Sns}{(S_n)_\nint{n}}
\newcommand{\ol}[1]{\overline{#1}}
\newcommand{\znz}{\Z/n\Z}
\renewcommand{\o}{\circ}

\newcommand{\seriegu}{\sum u_n}
\newcommand{\seriegv}{\sum v_n}
\newcommand{\harmonique}{\sum \frac{1}{n}}
\newcommand{\SRieman}{\sum \frac{1}{n^\alpha}}
\newcommand{\serie}[3]{\sum_{#1}^{#2}{#3}}
\newcommand{\satps}{série à terme positif}
\newcommand{\satp}{séries à termes positifs}
\newcommand{\pl}[1]{\mathbb{#1}[X]}
\newcommand{\som}[2]{\sum\limits_{#1}^{#2}}


\newcommand{\abs}[1]{\left\lvert#1\right\rvert}
\DeclarePairedDelimiter{\ceil}{\lceil}{\rceil}

%Format de fonctions:
\newcommand{\fct}[5]
	{
	  \begin{array}{ccccc}
		#1 & : & #2 & \to & #3 \\
	    && #4 & \mapsto & #5 \\
	  \end{array}
	}
\newcommand{\dfct}[2] {#1 \mapsto #2}

\renewcommand{\abs}[1]{|#1|}
\newcommand{\nm}[2]{ ||#1||_{#2} }
%===================TESTS===================

\begin{document}

\chapter{TD3}

\section{Question 1}

$\mathcal{F} = \lbrace (a,0) (b,0) \rbrace$\\
$\mathcal{P} = \lbrace (J,1),(A,2),(=,2) \rbrace$\\
$\mathcal{\epsilon} = \lbrace (\bot,0),(\top,0),(\neg,1),(\wedge,2),(\vee,2), (\to,2)  \rbrace$\\
$\mathcal{Q} = \lbrace \forall, \exists \rbrace$\\
Syntaxe:
\begin{itemize}
\item Termes $t := x|a|b ;  x \in X$
\item Formules $\Phi := J(t) |A(t,t)|t =t| \bot | \top | \neg \Phi | \Phi \wedge \Phi| \Phi \vee \Phi | \Phi \to \Phi | \forall x \Phi| \exists x \Phi$
\item $\mathcal{M} = (D ,I_c, I_v)$
\item $\mathcal{D} = \lbrace $individus $\rbrace $
\end{itemize}
\begin{itemize}
\item Bob n'apprécie que lui\\
		Réponse:$\forall x  ,(A(b,x) \Leftrightarrow b =x)$\\

\item Alice apprécie tout le monde,à l'exception de ceux qui portent des cravate jaunes\\
		Réponse:$\forall x  ,(\neg A(a,x) \Leftrightarrow J(x))$\\

\item Tous ceux qui portent des cravates jaunes s'apprécient mutuellement\\
Réponse:$\forall x ,y , (J(x) \wedge J(y)) \Rightarrow (A(x,y) \wedge A(y,x))$
\item Au plus trois personnes portes des cravates jaunes\\
		Réponse:$\exists x ,y ,z  (J(x) \wedge J(y) \wedge J(z)) \wedge (\forall  x_1 (\neg x_1 = x \wedge \neg x_1 = y \wedge \neg x_1 = z)\Rightarrow\neg J(x_1))$\\
		Réponse V2:   $ \forall x \forall y \forall z \forall t ( \neg (x = y) \wedge \neg ( x = z) \wedge \neg (y = z)) \wedge (J(x) \wedge J(y) \wedge J(z)) \Rightarrow ((t=x) \vee (t=y) \vee t =z))$
\item Au moins deux personnes portent des cravates jaunes\\
Réponse: $\exists x,y (J(x) \wedge J(y)) \wedge (\forall z (\neg A(x,z) \wedge \neg A(y,z))$
\item Il existe une personne qui apprécie quelqu 'un qui n'apprécie pas ceux qui portent des cravates jaunes\\
Réponse: $ \exists x ( \exists y ( A(x,y) \wedge (\forall w ( J(w) \Rightarrow \neg A(y,w)))))$
\end{itemize}

\section{Question 2}

\begin{enumerate}
\item $\forall x (P_1(x) \vee P_2(x) \vee P_3(x) \vee P_4(x) \vee P_5(x))$ : V
\item $\exists (P_1(x) \wedge P_2(x) \wedge P_3(x) \wedge P_4(x) \wedge P_5(x))$ :F $P_1$ et $P_5$ impossible
\item $\forall x P_1(x) \vee \forall x P_2(x) \vee \forall x P_3(x) \vee \forall x P_4(x) \vee \forall x P_5(x)$: F car : \begin{enumerate}
\item toujours impossible
\item pas vrais pour $x = 4$
\item pas vrais pour $x = 4$
\item pas vrais pour $x =1$
\item toujours impossible
\end{enumerate} 
\item $\exists P_1(x) \Rightarrow \exists P_5(x)$: V
\end{enumerate}

\section{Question 3}

Posons $\mathcal{I}_2$ d'univers $\mathcal{D}$ tel que :

$\forall j \in [0,8] ,\mathcal{I}_2(c_j) = 42$

$\forall i \in [1,5], \mathcal{I}_2(P_i) = \lbrace x \in \mathcal{D}  \rbrace$\\

vérifions les 4 formules
\begin{enumerate}
 \item  tous sont vérifiés 
 \item $x=0$ fonctionne
 \item vrais car $\forall x \in \mathcal{D} P_1(x) = V$
 \item vrais car $\forall x \in \mathcal{D} P_5(x) = V$
\end{enumerate}

\section{Question 4}

Posons $\mathcal{I}_2$ d'univers $\mathcal{D}$ tel que :

$\forall j \in [0,8] ,\mathcal{I}_2(c_j) = 42$

$\forall i \in [2,5], \mathcal{I}_2(P_i) = \emptyset$

$\mathcal{I}_2(P_1) = \lbrace 0 \rbrace$\\

vérifions les 4 formules
\begin{enumerate}
 \item  pas vrais pour tout $x$
 \item $P_2$ toujours faux
 \item toujours faux ou une seule valeur donc pas vrais por tout $x$
 \item bien un $x$ qui verifie $P_1$ et aucun qui verifie $P_5$ donc faux
\end{enumerate}

\section{Question 5}

Impossible car  si b est vrais il exite un $x$ tel que $P_1(x)$ est vrais et $P_5(x)$ donc b) est vraie or ce meme $x$ fais que d) est vrais 


\end{document}
