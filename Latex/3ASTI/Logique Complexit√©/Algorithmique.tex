\documentclass[12pt]{report}
\usepackage[utf8]{inputenc}
\usepackage[T1]{fontenc}
\usepackage[fleqn]{amsmath}
\usepackage{amsfonts,amssymb,stmaryrd}
\usepackage[english]{babel}
\usepackage{pdfpages}

%=============Affichage=======================
\usepackage{fullpage}
\usepackage{mathtools}
\usepackage{lmodern}
\usepackage{xcolor}
\usepackage{enumitem}
\usepackage{tikz,tkz-tab}
\usepackage[ruled,vlined]{algorithm2e}
\title{Cours Mathématiques MP*}
\author{Waxin Alban}

\definecolor{almond}{rgb}{0.94, 0.87, 0.8}
\definecolor{champagne}{rgb}{0.97, 0.91, 0.81}
\definecolor{dgreen}{rgb}{0.0, 0.5, 0.0}
\definecolor{bc}{rgb}{0.8588, 0.8980, 0.9450}

\setlength{\topmargin}{-1.5cm}
\setlength{\textheight}{25cm}
\setlength{\textwidth}{16cm}
\setlength{\oddsidemargin}{-1.5cm}
\setlength{\evensidemargin}{50cm}

\newcommand{\rd}[1]{\textcolor{red}{#1}}
\newcommand{\g}[1]{\textcolor{lime}{#1}}
\newcommand{\dg}[1]{\textcolor{dgreen}{#1}}
\newcommand{\blue}[1]{\textcolor{blue}{#1}}
\newcommand{\cy}[1]{\textcolor{cyan}{#1}}
\newcommand{\blz}{$\blacklozenge$}
\newcommand{\ns}{\\\indent\indent\vspace{0.25cm}}
\setcounter{secnumdepth}{5}% profondeur de la table des matières
\usepackage{titlesec}


\titleformat{\chapter}[frame]
{\Huge}
{\filright\rmfamily\bfseries\Huge\enspace\thechapter\enspace}
{18pt}
{\rmfamily\huge\bfseries\filcenter}
% rmfamily=roman, sffamily = sans serif ou ttfamily =type writer
\usepackage[many]{tcolorbox} % Creation de box collorable pour le texte non intégré
\newtcolorbox{mybox}{colback=bc,
colframe=black,arc=0mm,sharp corners= northwest,arc=10pt}

\newtcolorbox{demo}{colback=almond,
colframe=black,arc=0mm,sharp corners= northeast,arc=10pt}

\renewcommand*{\overrightarrow}[1]{\vbox{\halign{##\cr
 \tiny\rightarrowfill\cr\noalign{\nointerlineskip\vskip1pt}
 $#1\mskip2mu$\cr}}}

\newcommand{\rem}[1]
{
\subparagraph*{\underline{Remarque:#1}}\mbox{}\\
}

\newcommand{\props}[1]
{
\begin{mybox}
\textbf{\rd{\underline{\blz Propriété:} #1}}
\vspace{0.5cm}
\newline
}

\newcommand{\prope}
{
\end{mybox}
}

\newcommand{\scal}[2]
{
<#1|#2>
}

\newcommand{\defis}[1]
{
\begin{mybox}
\textbf{\rd{\underline{\blz Définition:} #1}}
\vspace{0.5cm}
\newline
}
\newcommand{\defie}
{
\end{mybox}
}
\newcommand{\demos}[1]
{
\begin{demo}
\textbf{\underline{\blz Démonstration:} #1}
\newline
}
\newcommand{\demoe}
{
\end{demo}
}
\newcommand{\exe}[1]
{
\subparagraph*{\underline{Exemple:#1}}\mbox{}\\
}

\newcommand{\vs}
{
\vspace{0.25cm}
}

\newcommand{\thms}[1]
{
\begin{mybox}
\textbf{\rd{\underline{\blz Théorème:} #1}}
\vspace{0.5cm}
\newline
}

\newcommand{\thme}
{
\end{mybox}
}

\newcommand{\coros}[1]
{
\begin{mybox}
\textbf{\rd{\underline{\blz Corolaire:} #1}}
\vspace{0.5cm}
\newline
}

\newcommand{\coroe}
{
\end{mybox}
}

\newcommand{\lems}[1]
{
\begin{mybox}
\textbf{\rd{\underline{\blz Lemme:} #1}}
\vspace{0.5cm}
\newline
}

\newcommand{\leme}
{
\end{mybox}
}
%=============================================

%\usepackage[cm]{aeguill}

%=============Mathématiques=================

%--------------Raccourcis:------------------
\newcommand{\R}{\mathbb{R}}
\newcommand{\C}{\mathbb{C}}
\newcommand{\N}{\mathbb{N}}
\newcommand{\Q}{\mathbb{Q}}
\newcommand{\Z}{\mathbb{Z}}
\newcommand{\K}{\mathbb{K}}
\newcommand{\M}{\mathcal{M}}
\newcommand{\nint}[1]{#1 \in \N}
\newcommand{\zint}[1]{#1 \in \N^*}
\newcommand{\limi}[1]{\underset{#1 \to \infty}{lim}}
\newcommand{\limn}[2]{\underset{#1 \to #2}{lim}}
\newcommand{\x}{\times}
\newcommand{\un}[1]{u_{#1}}
\newcommand{\uns}{(u_n)_\nint{n}}
\newcommand{\Sn}[1]{S_{#1}}
\newcommand{\Sns}{(S_n)_\nint{n}}
\newcommand{\ol}[1]{\overline{#1}}
\newcommand{\znz}{\Z/n\Z}
\renewcommand{\o}{\circ}

\newcommand{\seriegu}{\sum u_n}
\newcommand{\seriegv}{\sum v_n}
\newcommand{\harmonique}{\sum \frac{1}{n}}
\newcommand{\SRieman}{\sum \frac{1}{n^\alpha}}
\newcommand{\serie}[3]{\sum_{#1}^{#2}{#3}}
\newcommand{\satps}{série à terme positif}
\newcommand{\satp}{séries à termes positifs}
\newcommand{\pl}[1]{\mathbb{#1}[X]}
\newcommand{\som}[2]{\sum\limits_{#1}^{#2}}


\newcommand{\abs}[1]{\left\lvert#1\right\rvert}
\DeclarePairedDelimiter{\ceil}{\lceil}{\rceil}

%Format de fonctions:
\newcommand{\fct}[5]
	{
	  \begin{array}{ccccc}
		#1 & : & #2 & \to & #3 \\
	    && #4 & \mapsto & #5 \\
	  \end{array}
	}
\newcommand{\dfct}[2] {#1 \mapsto #2}

\renewcommand{\abs}[1]{|#1|}
\newcommand{\nm}[2]{ ||#1||_{#2} }
%===================TESTS===================

\begin{document}

\chapter{}

\section{Tri par insertion}

\begin{center}
    \scalebox{0.75}{
    
    \begin{minipage}{\linewidth}
\begin{algorithm}[H]
\SetAlgoLined
	\KwData{Un tableau A de longueur n}
	\KwResult{Un tableau trié}
	\BlankLine
	 int $i,j,clé$ \;
	\For{$j = 2, j \leq n,j++$}
	{
	$clé \gets A[j]$ \;
	$i \gets j-1$ \;
	 \While{($i > 0$ and $A[i] > clé$)}
	 	{
	 	  $A[i+1] \gets A[i]$ \;
	 	  $i \gets i-1$ \;
	 	}
	 $A[i+] \gets clé$ \;
	}
		
\caption{Pseudo code du tri par insertion}
\end{algorithm}
\end{minipage}%
    }
\end{center}

\paragraph*{Invariants de boucle}

$A_0$: tableau initial\\
$A_j$: tableau en mémoire au début de l'itération $j$ de la boucle \verb|for|\\
$A$: tableau en mémoire à l'instant courant\\
$A[a...b]$:  le sous tableau contenant les éléments dont l'indice $k$ est $a \leq k \leq b$

\begin{enumerate}
	\item $A[j+1 \to n] = A_j[j+1 \to n]$
	\item $A[1 \to i] = A_j[1 \to i]$
	\item $A[i+2 \to j] =A_j[i+1 \to j-1]$
	\item $clé = A_j[j]$
	\item $A[i+2 \to j] > clé$
\end{enumerate}

\paragraph*{Initialisation}
Rq: $A[]$ : tableau vide
\begin{enumerate}
	\item $A[j+1 \to n] = A_j[j+1 \to n]$ oui car on n'a pas modifié le tableau
	\item $A[1 \to i] = A_j[1 \to i]$ idem
	\item $A[i+2 \to j] = \underbrace{A_j[j+1 \to j]}_{tableau vide} = \underbrace{A_j[j \to j-1]}_{tableau vide}$ 
	\item vide
	\item $A[i+2 \to j]$: tableau vide donc tous les éléments du tableau sont >clé
\end{enumerate}

\paragraph*{Conservation}
On considère l'itération $j = k$\\
On suppose les propriétés 1,2,3,4,5 vraies au début de l'itération\\
\subparagraph*{Objectifs} prouver que les propriétés sont vraies au début de l'itéartion suivante\\
On prouve la préservation donc on suppose qu'on est rentré dans la boucle \verb|while| et $i > 0$ et $A[i] > clé$ 
\begin{enumerate}
 \item vrai car clé non modifié
 \item i est initialisé à $j -1$\\
 toujours décrémenté\\
 et on modifie la case i+1 qui est donc d'indice  $\leq j$\\
 donc le tableau $A[j+1 \to n]$ n'est pas modifié
 \item ligne 6 la case d'indice $i+1$ est modifiée\\
 ligne 7 $i \gets i-1$ donc avec la nouvelle valeur de $i$ c est la case $A[i+2]$ qui a été modifiée \\
 Donc les cases $A[1 \to i]$ ne sont pas modifiées
 \item[5] ligne 5 $\underbrace{A[i+2 \to j]}_{> clé} = A_j[i+1 \to j-1]$\\
 $A[i] > clé$ ( test du while)\\
 ligne 6 $A[i+1] \gets A[i] = A_j[i]$\\
 Donc $A[i+1 ... j] =  A_j [i \to j-1]$\\
 ligne 7 $i \gets i-1$\\
 Donc $A[i+2 \to j ] = A_j[i+1 \to j-1]$ 

\end{enumerate}

\subsection*{Compléxité}

\begin{align*}
T(n)  &= ac + \sum_{j=2}^n (bc +t_j)\\
&= ac + \sum_{j=2}^nbc + \sum_{j=2}^n t_j\\
&= (a-b)c + nbc + \sum_{j=2}^n t_j
\end{align*}

\subparagraph*{meilleur des cas: tableau trié}
Lors de la $1^{ère}$ itération de la boucle while\\
le test $A[i] > clé$ est faux\\
donc $t_j = \beta$ (temps  constant)\\
$T_M(n) =  \Theta(n)$\\

\subparagraph*{Pire des cas}
$T_P(n) = O(n^2)$

\chapter{Complexité}

\end{document}
