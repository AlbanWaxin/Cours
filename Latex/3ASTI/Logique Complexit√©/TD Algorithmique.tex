\documentclass[12pt]{report}
\usepackage[utf8]{inputenc}
\usepackage[T1]{fontenc}
\usepackage[fleqn]{amsmath}
\usepackage{amsfonts,amssymb,stmaryrd}
\usepackage[english]{babel}
\usepackage{pdfpages}

%=============Affichage=======================
\usepackage{fullpage}
\usepackage{mathtools}
\usepackage{lmodern}
\usepackage{xcolor}
\usepackage{enumitem}
\usepackage{tikz,tkz-tab}
\usepackage[ruled,vlined]{algorithm2e}
\title{Exercice Algorithmique et compléxité}
\author{Waxin Alban}

\definecolor{almond}{rgb}{0.94, 0.87, 0.8}
\definecolor{champagne}{rgb}{0.97, 0.91, 0.81}
\definecolor{dgreen}{rgb}{0.0, 0.5, 0.0}
\definecolor{bc}{rgb}{0.8588, 0.8980, 0.9450}

\setlength{\topmargin}{-1.5cm}
\setlength{\textheight}{25cm}
\setlength{\textwidth}{16cm}
\setlength{\oddsidemargin}{-1.5cm}
\setlength{\evensidemargin}{50cm}

\newcommand{\rd}[1]{\textcolor{red}{#1}}
\newcommand{\g}[1]{\textcolor{lime}{#1}}
\newcommand{\dg}[1]{\textcolor{dgreen}{#1}}
\newcommand{\blue}[1]{\textcolor{blue}{#1}}
\newcommand{\cy}[1]{\textcolor{cyan}{#1}}
\newcommand{\blz}{$\blacklozenge$}
\newcommand{\ns}{\\\indent\indent\vspace{0.25cm}}
\setcounter{secnumdepth}{5}% profondeur de la table des matières
\usepackage{titlesec}


\titleformat{\chapter}[frame]
{\Huge}
{\filright\rmfamily\bfseries\Huge\enspace\thechapter\enspace}
{18pt}
{\rmfamily\huge\bfseries\filcenter}
% rmfamily=roman, sffamily = sans serif ou ttfamily =type writer
\usepackage[many]{tcolorbox} % Creation de box collorable pour le texte non intégré
\newtcolorbox{mybox}{colback=bc,
colframe=black,arc=0mm,sharp corners= northwest,arc=10pt}

\newtcolorbox{demo}{colback=almond,
colframe=black,arc=0mm,sharp corners= northeast,arc=10pt}

\renewcommand*{\overrightarrow}[1]{\vbox{\halign{##\cr
 \tiny\rightarrowfill\cr\noalign{\nointerlineskip\vskip1pt}
 $#1\mskip2mu$\cr}}}

\newcommand{\rem}[1]
{
\subparagraph*{\underline{Remarque:#1}}\mbox{}\\
}

\newcommand{\props}[1]
{
\begin{mybox}
\textbf{\rd{\underline{\blz Propriété:} #1}}
\vspace{0.5cm}
\newline
}

\newcommand{\prope}
{
\end{mybox}
}

\newcommand{\scal}[2]
{
<#1|#2>
}

\newcommand{\defis}[1]
{
\begin{mybox}
\textbf{\rd{\underline{\blz Définition:} #1}}
\vspace{0.5cm}
\newline
}
\newcommand{\defie}
{
\end{mybox}
}
\newcommand{\demos}[1]
{
\begin{demo}
\textbf{\underline{\blz Démonstration:} #1}
\newline
}
\newcommand{\demoe}
{
\end{demo}
}
\newcommand{\exe}[1]
{
\subparagraph*{\underline{Exemple:#1}}\mbox{}\\
}

\newcommand{\vs}
{
\vspace{0.25cm}
}

\newcommand{\thms}[1]
{
\begin{mybox}
\textbf{\rd{\underline{\blz Théorème:} #1}}
\vspace{0.5cm}
\newline
}

\newcommand{\thme}
{
\end{mybox}
}

\newcommand{\coros}[1]
{
\begin{mybox}
\textbf{\rd{\underline{\blz Corolaire:} #1}}
\vspace{0.5cm}
\newline
}

\newcommand{\coroe}
{
\end{mybox}
}

\newcommand{\lems}[1]
{
\begin{mybox}
\textbf{\rd{\underline{\blz Lemme:} #1}}
\vspace{0.5cm}
\newline
}

\newcommand{\leme}
{
\end{mybox}
}
%=============================================

%\usepackage[cm]{aeguill}

%=============Mathématiques=================

%--------------Raccourcis:------------------
\newcommand{\R}{\mathbb{R}}
\newcommand{\C}{\mathbb{C}}
\newcommand{\N}{\mathbb{N}}
\newcommand{\Q}{\mathbb{Q}}
\newcommand{\Z}{\mathbb{Z}}
\newcommand{\K}{\mathbb{K}}
\newcommand{\M}{\mathcal{M}}
\newcommand{\nint}[1]{#1 \in \N}
\newcommand{\zint}[1]{#1 \in \N^*}
\newcommand{\limi}[1]{\underset{#1 \to \infty}{lim}}
\newcommand{\limn}[2]{\underset{#1 \to #2}{lim}}
\newcommand{\x}{\times}
\newcommand{\un}[1]{u_{#1}}
\newcommand{\uns}{(u_n)_\nint{n}}
\newcommand{\Sn}[1]{S_{#1}}
\newcommand{\Sns}{(S_n)_\nint{n}}
\newcommand{\ol}[1]{\overline{#1}}
\newcommand{\znz}{\Z/n\Z}
\renewcommand{\o}{\circ}

\newcommand{\seriegu}{\sum u_n}
\newcommand{\seriegv}{\sum v_n}
\newcommand{\harmonique}{\sum \frac{1}{n}}
\newcommand{\SRieman}{\sum \frac{1}{n^\alpha}}
\newcommand{\serie}[3]{\sum_{#1}^{#2}{#3}}
\newcommand{\satps}{série à terme positif}
\newcommand{\satp}{séries à termes positifs}
\newcommand{\pl}[1]{\mathbb{#1}[X]}
\newcommand{\som}[2]{\sum\limits_{#1}^{#2}}


\newcommand{\abs}[1]{\left\lvert#1\right\rvert}
\DeclarePairedDelimiter{\ceil}{\lceil}{\rceil}

%Format de fonctions:
\newcommand{\fct}[5]
	{
	  \begin{array}{ccccc}
		#1 & : & #2 & \to & #3 \\
	    && #4 & \mapsto & #5 \\
	  \end{array}
	}
\newcommand{\dfct}[2] {#1 \mapsto #2}

\renewcommand{\abs}[1]{|#1|}
\newcommand{\nm}[2]{ ||#1||_{#2} }
%===================TESTS===================

\begin{document}
\maketitle

\chapter*{Exercice:}

Soit $a > 0$\\
Soient $k,k' \in \N²$\\

\begin{enumerate}
\item Pourquoi \begin{align*}
a n^k &= \Theta(n^k)\\
&= O(n^k)\\
&= \Omega(n^k)
\end{align*}
\item Pourquoi \begin{align*}
an^k &=  O(n^{k'}) \text{avec} k' \geq k\\
&= \Omega(n^{k'}) \text{avec} k' \leq k
\end{align*}
\end{enumerate}
Réponse
\begin{enumerate}
\item \begin{itemize}
\item Par définition de $\Theta$, $\exists C,D > 0, \exists n_0, \forall n > n_0 0 \leq Cg(n) \leq f(n) \leq Dg(n)$\\
En posant C = a et D = a on a bien $\frac{a n^k}{n^k}  = a$ et $C \leq a \leq D$
\item Par définition $f= \Theta g \Leftrightarrow f = O(g) \wedge f = \Omega(g)$
\end{itemize} 
\item \begin{itemize}
\item Par définition de O, $\exists C > 0, \exists n_0, \forall n > n_0 0 \leq f(n) \leq Cg(n)$\\
Or $\frac{an^k}{n^{k'}} = an^{k-k'}$ avec $ k-k' < 0$\\
D'ou $\frac{an^k}{n^{k'}} \to_{\infty} 0$ \\
donc $an^k =  O(n^{k'}) \text{avec} k' \geq k$ 
\item  De même $\frac{an^k}{n^{k'}} = an^{k-k'}$ avec $ k-k' > 0$\\
La fonction est donc strictement croissante vers l'infini elle est donc plus grande qu un seuil qui est le cas $n=1$ et $C=a$ donc Dans la définition de $\Omega$ avec C = a on a bien $ an^k= \Omega(n^{k'}) \text{avec} k' \leq k$
\end{itemize}
\end{enumerate}

\chapter{TD}

\section{}
\section{Compléxité}

\subsection*{Exercice 46}

\begin{center}
    \scalebox{0.75}{
    
    \begin{minipage}{\linewidth}
\begin{algorithm}[H]
\SetAlgoLined
	\KwData{Deux Matrices A,B de taille n}
	\KwResult{Une matrice C du produit}
	\BlankLine
	 int $i,j,k$ \;
	\For{$i = 1, i \leq n,i++$}
	{
	\For{$j= 1, j \leq n, j++$}
	 	{
	 	  int $s = 0$ \;
	 	  \For{$k =1, k \leq n, k++$}
	 	  {
	 	    $s += A_{i,k} \times B_{k,j}$
	 	  }
	 	  $C_{i,j} \gets s$
	 	}
	}
		
\caption{Produit Matriciel}
\end{algorithm}
\end{minipage}%
    }
\end{center}

\paragraph*{Compléxité:}
La complexite de l'algorithme est $O(n^3)$:\\
cout boucle for en $i$, $c_i = n * c_j$\\
cout boucle for en $i$, $c_j = n * c_k$\\
cout boucle for en $k$, $c_k = a*n$\\
Cout total: $n^3$

\paragraph*{Correction:}

\subparagraph*{Etude de la boucle For en k}

Fonction: la boucle For calcul bien la valeur de la case $C_{i,j}$

Notation:

\begin{itemize}
\item $s_0$  est le compteur au depart
\item $s_k$  est le compteur au début de k tours de boucle
\item $s$ est le compteur au moment présent 
\item $C_k$ est la matrice du produit
\end{itemize}

La boucle for en k  effectue les opérations suivantes:

\begin{itemize}
\item En entrée de la boucle en $k$, on a : $s = s_{k-1}$
\item A la fin de chaque itération , on a : $s = s_{k-1} + A_{i,k}B_{k,j}$
\item En sortie de la boucle en k :$ s = \sum_1^n A_{i,k}B_{k,j}$
\end{itemize}

\subparagraph*{invariants de boucle}

\begin{enumerate}
 \item $s_x = \sum_1^x  A_{i,k}B_{k,j}$
 \item $ A = A_k, B = B_k$
 \item $ C = C_k$
\end{enumerate}

\subparagraph*{Preuve:}

\begin{itemize}
\item Initialisation:\\
\begin{enumerate}
\item Avant la premiere itération s = 0
\end{enumerate}
\end{itemize}


\end{document}