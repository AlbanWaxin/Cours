\documentclass[12pt]{report}
\usepackage[utf8]{inputenc}
\usepackage[T1]{fontenc}
\usepackage[fleqn]{amsmath}
\usepackage{amsfonts,amssymb,stmaryrd}
\usepackage[english]{babel}
\usepackage{pdfpages}

%=============Affichage=======================
\usepackage{fullpage}
\usepackage{mathtools}
\usepackage{lmodern}
\usepackage{xcolor}
\usepackage{enumitem}
\usepackage{tikz,tkz-tab}
\title{Cours Mathématiques MP*}
\author{Waxin Alban}

\definecolor{almond}{rgb}{0.94, 0.87, 0.8}
\definecolor{champagne}{rgb}{0.97, 0.91, 0.81}
\definecolor{dgreen}{rgb}{0.0, 0.5, 0.0}
\definecolor{bc}{rgb}{0.8588, 0.8980, 0.9450}

\setlength{\topmargin}{-1.5cm}
\setlength{\textheight}{25cm}
\setlength{\textwidth}{16cm}
\setlength{\oddsidemargin}{-1.5cm}
\setlength{\evensidemargin}{50cm}

\newcommand{\rd}[1]{\textcolor{red}{#1}}
\newcommand{\g}[1]{\textcolor{lime}{#1}}
\newcommand{\dg}[1]{\textcolor{dgreen}{#1}}
\newcommand{\blue}[1]{\textcolor{blue}{#1}}
\newcommand{\cy}[1]{\textcolor{cyan}{#1}}
\newcommand{\blz}{$\blacklozenge$}
\newcommand{\ns}{\\\indent\indent\vspace{0.25cm}}
\setcounter{secnumdepth}{5}% profondeur de la table des matières
\usepackage{titlesec}


\titleformat{\chapter}[frame]
{\Huge}
{\filright\rmfamily\bfseries\Huge\enspace\thechapter\enspace}
{18pt}
{\rmfamily\huge\bfseries\filcenter}
% rmfamily=roman, sffamily = sans serif ou ttfamily =type writer
\usepackage[many]{tcolorbox} % Creation de box collorable pour le texte non intégré
\newtcolorbox{mybox}{colback=bc,
colframe=black,arc=0mm,sharp corners= northwest,arc=10pt}

\newtcolorbox{demo}{colback=almond,
colframe=black,arc=0mm,sharp corners= northeast,arc=10pt}

\renewcommand*{\overrightarrow}[1]{\vbox{\halign{##\cr
 \tiny\rightarrowfill\cr\noalign{\nointerlineskip\vskip1pt}
 $#1\mskip2mu$\cr}}}

\newcommand{\rem}[1]
{
\subparagraph*{\underline{Remarque:#1}}\mbox{}\\
}

\newcommand{\props}[1]
{
\begin{mybox}
\textbf{\rd{\underline{\blz Propriété:} #1}}
\vspace{0.5cm}
\newline
}

\newcommand{\prope}
{
\end{mybox}
}

\newcommand{\scal}[2]
{
<#1|#2>
}

\newcommand{\defis}[1]
{
\begin{mybox}
\textbf{\rd{\underline{\blz Définition:} #1}}
\vspace{0.5cm}
\newline
}
\newcommand{\defie}
{
\end{mybox}
}
\newcommand{\demos}[1]
{
\begin{demo}
\textbf{\underline{\blz Démonstration:} #1}
\newline
}
\newcommand{\demoe}
{
\end{demo}
}
\newcommand{\exe}[1]
{
\subparagraph*{\underline{Exemple:#1}}\mbox{}\\
}

\newcommand{\vs}
{
\vspace{0.25cm}
}

\newcommand{\thms}[1]
{
\begin{mybox}
\textbf{\rd{\underline{\blz Théorème:} #1}}
\vspace{0.5cm}
\newline
}

\newcommand{\thme}
{
\end{mybox}
}

\newcommand{\coros}[1]
{
\begin{mybox}
\textbf{\rd{\underline{\blz Corolaire:} #1}}
\vspace{0.5cm}
\newline
}

\newcommand{\coroe}
{
\end{mybox}
}

\newcommand{\lems}[1]
{
\begin{mybox}
\textbf{\rd{\underline{\blz Lemme:} #1}}
\vspace{0.5cm}
\newline
}

\newcommand{\leme}
{
\end{mybox}
}
%=============================================

%\usepackage[cm]{aeguill}

%=============Mathématiques=================

%--------------Raccourcis:------------------
\newcommand{\R}{\mathbb{R}}
\newcommand{\C}{\mathbb{C}}
\newcommand{\N}{\mathbb{N}}
\newcommand{\Q}{\mathbb{Q}}
\newcommand{\Z}{\mathbb{Z}}
\newcommand{\K}{\mathbb{K}}
\newcommand{\M}{\mathcal{M}}
\newcommand{\nint}[1]{#1 \in \N}
\newcommand{\zint}[1]{#1 \in \N^*}
\newcommand{\limi}[1]{\underset{#1 \to \infty}{lim}}
\newcommand{\limn}[2]{\underset{#1 \to #2}{lim}}
\newcommand{\x}{\times}
\newcommand{\un}[1]{u_{#1}}
\newcommand{\uns}{(u_n)_\nint{n}}
\newcommand{\Sn}[1]{S_{#1}}
\newcommand{\Sns}{(S_n)_\nint{n}}
\newcommand{\ol}[1]{\overline{#1}}
\newcommand{\znz}{\Z/n\Z}
\renewcommand{\o}{\circ}

\newcommand{\seriegu}{\sum u_n}
\newcommand{\seriegv}{\sum v_n}
\newcommand{\harmonique}{\sum \frac{1}{n}}
\newcommand{\SRieman}{\sum \frac{1}{n^\alpha}}
\newcommand{\serie}[3]{\sum_{#1}^{#2}{#3}}
\newcommand{\satps}{série à terme positif}
\newcommand{\satp}{séries à termes positifs}
\newcommand{\pl}[1]{\mathbb{#1}[X]}
\newcommand{\som}[2]{\sum\limits_{#1}^{#2}}


\newcommand{\abs}[1]{\left\lvert#1\right\rvert}
\DeclarePairedDelimiter{\ceil}{\lceil}{\rceil}

%Format de fonctions:
\newcommand{\fct}[5]
	{
	  \begin{array}{ccccc}
		#1 & : & #2 & \to & #3 \\
	    && #4 & \mapsto & #5 \\
	  \end{array}
	}
\newcommand{\dfct}[2] {#1 \mapsto #2}

\renewcommand{\abs}[1]{|#1|}
\newcommand{\nm}[2]{ ||#1||_{#2} }
%===================TESTS===================

\begin{document}

\chapter{TD1}

\section{Exercice 1}

\paragraph*{Question 1)}

\paragraph*{Question Bonus:}
On décide le nombre de broche dans le bus d'adresse  par deux manières \\
Premièrement en comptant le nombre de broche de controle soit ici 10 (8 bits de données et 2 bits de controle A et B)\\
Ou en comptant par rapport a la taille des mémoires ici $256 * 4 = 1024= 2^10$ on a donc besoin de 10 bits de controle
\section{Exercice 2}

\paragraph*{Question 1)} 
 Le circuit considéré utilise 8 circuits de RAM de $256 \times 4$ bits et un couple utilise deux de ces circuits en parallèle ce qui ne change pas la capacite mais double la longueur des mots soit $256\times 8$ bits
 
 \paragraph*{Questions 2)}
 
Les équations de validation des couples dépend des tables de vérité du multiplexeurs, les equations suivantes dépendent de la table de vérité suivantes

\begin{tabular}{|c|c|c|c|}
\hline
A&B&C&S\\
\hline
0&0&0&0\\
0&0&1&1\\
0&1&0&2\\
0&1&1&3\\
1&0&0&4\\
1&0&1&5\\
1&1&0&6\\
1&1&1&7\\
\hline
\end{tabular}

\begin{itemize}
\item 1/2: $0 =  (A = 0) \wedge (B = 0) \wedge (C=0) $
$= \neg A_8 \wedge \neg A_9 \wedge \neg (\bigwedge_{i \in [10,15]} \neg A_i)$ 
\item 3/4: $1 =  (A = 0) \wedge (B = 0) \wedge (C=1)$
$= \neg A_8 \wedge \neg A_9 \wedge  (\bigwedge_{i \in [10,15]} \neg A_i)$ 
\item 5/6: $2 =  (A = 0) \wedge (B = 1) \wedge (C=0)$
$= \neg A_8 \wedge  A_9 \wedge \neg (\bigwedge_{i \in [10,15]} \neg A_i)$ 
\item 7/8: $0 =  (A = 0) \wedge (B = 1) \wedge (C=1)$
$= \neg A_8 \wedge  A_9 \wedge  (\bigwedge_{i \in [10,15]} \neg A_i)$ 
\end{itemize} 

\paragraph*{Question 3)}
Par résultat de la question 1) et par mise en série des blocs mémoires (controlable par des broches différentes) alors la taille finale de l'ensemble est $(4 \times 256) \times 8$ bits  soit $1024 \times 8$ bits

\chapter{TD2}

\section{Exercice 1}

Contexte: Programme de 48 instructions à répéter 2 fois.\\

Les adresses mémoire des instructions sont $\boxed{1 \to 8}\boxed{145 \to 160}\boxed{9 \to 16}\boxed{145 \to 160}$

\paragraph*{Question 1)}
On considère un cache de 3 blocs de 8 instructions:
On regarde le cache par tour de Remplissage
\begin{itemize}
\item Tour 1:
\item $\underbrace{\boxed{\phantom{555} 1 \to 8 \phantom{555}}}_{1M +7C}\boxed{\phantom{55555555555}}\boxed{\phantom{55555555555}}$
\item $\boxed{\phantom{555} 1 \to 8 \phantom{555}}\underbrace{\boxed{\phantom{5} 145  \to 152 \phantom{5}}}_{1M +7C} \underbrace{\boxed{\phantom{5} 153  \to 160 \phantom{5}}}_{1M +7C}$
\item $\underbrace{\boxed{\phantom{555} 9 \to 16 \phantom{555}}}_{1M +7C}\boxed{\phantom{5} 145  \to 152 \phantom{5}} \boxed{\phantom{5} 153  \to 160 \phantom{5}}$
\item $\boxed{\phantom{555} 9 \to 16 \phantom{555}}\underbrace{\boxed{\phantom{5} 145  \to 152 \phantom{5}}}_{8C} \underbrace{\boxed{\phantom{5} 153  \to 160 \phantom{5}}}_{8C}$
\item Tour 2:
\item $\underbrace{\boxed{\phantom{555} 1 \to 8 \phantom{555}}}_{1M +7C}\boxed{\phantom{5} 145  \to 152 \phantom{5}} \boxed{\phantom{5} 153  \to 160 \phantom{5}}$
\item $\boxed{\phantom{555} 1 \to 8 \phantom{555}}\underbrace{\boxed{\phantom{5} 145  \to 152 \phantom{5}}}_{8C} \underbrace{\boxed{\phantom{5} 153  \to 160 \phantom{5}}}_{8C}$
\item $\underbrace{\boxed{\phantom{555} 9 \to 16 \phantom{555}}}_{1M +7C}\boxed{\phantom{5} 145  \to 152 \phantom{5}} \boxed{\phantom{5} 153  \to 160 \phantom{5}}$
\item $\boxed{\phantom{555} 9 \to 16 \phantom{555}}\underbrace{\boxed{\phantom{5} 145  \to 152 \phantom{5}}}_{8C} \underbrace{\boxed{\phantom{5} 153  \to 160 \phantom{5}}}_{8C}$
\end{itemize}

Calcul du total: $(1M + 7C) \times 4 + (8C \times 2) + 1M + 7C + (8C \times 2) + 1M + 7C + (8C \times 2) $ \\
$= 6M +  90C$

\paragraph*{Question 2)}
On considère un cache de 2 blocs de 12 instructions:
On regarde le cache par tour de Remplissage
\begin{itemize}
\item Tour 1:
\item $\underbrace{\boxed{\phantom{555} 1 \to 12 \phantom{555}}}_{1M +7C}\boxed{\phantom{55555555555}}$  7C car même si on regarde le bloc $1 \to 12$ seul $1 \to 8$ sont regardés en cache
\item $\boxed{\phantom{555} 1 \to 12 \phantom{555}}\underbrace{\boxed{\phantom{5} 145  \to 156 \phantom{5}}}_{1M +11C}$
\item $\underbrace{\boxed{\phantom{555} 157 \to 168 \phantom{555}}}_{1M +3C}\boxed{\phantom{5} 145  \to 156 \phantom{5}}$ 3C car même si on regarde le bloc $157 \to 168$ seul $157 \to 160$ sont regardés en cache
\item $\boxed{\phantom{555} 157 \to 168 \phantom{555}}\underbrace{\boxed{\phantom{5} 1  \to 12 \phantom{5}}}_{1M +3C}$
\item $\underbrace{\boxed{\phantom{5} 13  \to 24 \phantom{5}}}_{1M +3C}\boxed{\phantom{5} 1  \to 12 \phantom{5}}$
\item $\underbrace{\boxed{\phantom{555} 145 \to 156 \phantom{555}}}_{1M +11C}\underbrace{\boxed{\phantom{5} 157  \to 168 \phantom{5}}}_{1M + 3C}$
\item Tour 2: Le tour 2 consomme exactement comme le tour 1 car au debut du tour 2 aucun des element du cache va etre réutilisé avant d'être écrit par dessus
\end{itemize}

Calcul du total: $2 \times ((1M + 7C) + (1M +11C) + ((1M +3C) \times 3) + (1M + 11C) + (1M +3C)$ \\
$=2 \times  (7M +  41C) = 14M + 82C$

\paragraph*{Question 1)}
On considère un cache de 1 blocs de 24 instructions:
On regarde le cache par tour de Remplissage
\begin{itemize}
\item Tour 1:
\item $\underbrace{\boxed{\phantom{555} 1 \to 24 \phantom{555}}}_{1M +7C}$ 7C car même si on regarde le bloc $1 \to 24$ seul $1 \to 8$ sont regardés en cache
\item $\underbrace{\boxed{\phantom{555} 145 \to 168 \phantom{555}}}_{1M +15C}$
\item $\underbrace{\boxed{\phantom{555} 1 \to 24 \phantom{555}}}_{1M +7C}$ 7C car même si on regarde le bloc $1 \to 24$ seul $9 \to 16$ sont regardés en cache
\item $\underbrace{\boxed{\phantom{555} 145 \to 168 \phantom{555}}}_{1M +15C}$
\item Tour 2: Le tour 2 consomme exactement comme le tour 1 car au debut du tour 2 aucun des element du cache va etre réutilisé avant d'être écrit par dessus
\end{itemize}

Calcul du total: $2 \times (1M +7C) + (1M + 15C) + (1M + 7C) + (1M+15C)$ \\
$=2 \times  (4M + 44C) = 8M + 88C$ \\

\paragraph*{Question Bonus: Comparaison}
Entre le format du cache de la question 1) et le format de cache de la question 3) la question qui présente le format le plus efficace est le format 3 blocs de 8 instructions  car :\\
$6 M + 90C  \sim 6M$ ( car $M \gg C$) et $8M + 88C \sim 8M$
Or $6 < 8 $
Donc le plus efficace est la question 1) 
\section{Exercice 2}

Rappel de la formule de l'exercice:\\
 $t_i = t_{acces-succes} \times taux_succes + (temps_{acces_echec} + t_{i+1}) \times taux_{echec}$
 \ns 
 
 Définition de la règle du $t_{i+1}$: Lors du calcul du temps pour un niveau de mémoire  le $t_{i+1}$ représente le temps du niveau de mémoire suivant dans la liste des distances vers la memoire principale.\\
 Exemples:
 \begin{itemize}
 	\item question 1): pas de mémoire au delà de la mémoire principale
 	\item question 2): pour L1 $t_{i+1}$ est $t$ de  la mémoire principale
 	\item question 3): pour L1 $t_{i+1}$ est  $t$ de L2\\
 									  pour L2 $t_{i+1}$ est $t$ de la mémoire principale
 	\item question 4) pour L1 $t_{i+1}$  est $t$ de L2\\
 									  pour L2 $t_{i+1}$ est $t$ de L3\\
 									  pour L3 $t_{i+1}$ est $t$ de la mémoire principale 
 \end{itemize}
 
 \paragraph*{Question 1)}
 En appliquant la formule avec un taux de succès de 1 pour la mémoire principale : $ t_M = 40$
 
 \paragraph*{Question 2)}
 On applique d'abord la formule pour la mémoire principale\\
 $t_M = 40$\\
 Ensuite pour L1, $t_1 =  3 *0.8 + (1 + 40)*0.2 = \boxed{10.6}$\\

 \paragraph*{Question 3)}
 On applique d'abord la formule pour la mémoire principale\\
 $t_M = 40$\\
 Ensuite pour L2, $t_2 =  5 *0.9 + (2 + 40)*0.1 = \boxed{8.7}$\\
 Enfin pour L1, $t_1 =  3 *0.8 + (1 + 8.7)*0.2 = \boxed{4.34}$\\
 
 \paragraph*{Question 4)}
 On applique d'abord la formule pour la mémoire principale\\
 $t_M = 40$\\
 Ensuite pour L3, $t_3 =  12 *0.95 + (4 + 40)*0.05 = \boxed{13.6}$\\
 Ensuite pour L2, $t_2 =  5 *0.9 + (2 + 13.6)*0.1 = \boxed{6.06}$\\
 Enfin pour L1, $t_1 =  3 *0.8 + (1 + 6.06)*0.2 = \boxed{3.812}$\\
 
\section{Exercice 3}

\paragraph*{Question 1)}

Les unité de consommation en $\mu c$ (micro cycles), sachant que $1 \mu c = 1 ns$ ($ns = 10^{-9} secondes$) 
Or entre chaque solicitation, une solicitation étant une nouvelle tache donné par le programme, il y a un délai de $15 ns$\\

La première étape hors boucle consome alors $d = 10$ (MOV) $ + 2$(MOV)  = \underline{$12 ns$}  (Pas de délai entre les opérations car les opérations de mouvements sont des opérations élémentaires)\\
\ns
De plus une tour de boucle consomme  $b=17$ (Décrémentation) $ +15$ (Délai) $+16$ (Boucle) $=48 ns$ \\
\ns
Or Il y a 0100H tours de boucle soit 256 tours, D'ou $t_b = b *256 =8 448 ns$ \\
\ns
La dernière étape hors boucle consome alors $f= 4$ (MOV) $+52$ (INT)$=56 \mu  c$\\

Donc au total un temps de calcul de : $\boxed{12356 ns}$
\vspace{1.5cm}

Pour obtenir la valeur en terme de cycle on ne considère plus les temps d'appel donc $b_2 = 33\mu c$

Le total est donc  de $\boxed{8516 \mu c}$ 

\paragraph*{Question 2)}

On suppose maintenant que le processeur dispose d'une mémoire cache 

Comme la mémoire cache stock les actions de la boucle il n'y a plus d'appel mémoire à chaque itération mais on fais 15 appels à la mémoire cache
donc  on reprend la réponse en $\mu c$ et on lui ajoute $15 ns$\\
\ns
Soit un total de $\boxed{8531 ns}$ \\
\ns 

une autre réponse possible est lorsque l'on considère que les couts de processeur ne sont pas précomptées dans le calcul on rajoute alors 255 ms 
donc un total de$\boxed{8786 ns}$


\end{document}
