\documentclass{beamer}
\usetheme{Boadilla}
\usecolortheme{beaver}
\usepackage[utf8]{inputenc}
\usepackage[T1]{fontenc}
\usepackage{amsmath,amsfonts,amssymb,stmaryrd} 
\usepackage[english]{babel}
\usepackage{minted}
\usepackage{xcolor}
\usepackage{color}
\usepackage{tcolorbox}
\tcbuselibrary{minted,skins}
%\usemintedstyle{monokai}
%\definecolor{bg}{rgb}{0.25,0.25,0.25}
%\newminted[ocaml2]{Ocaml}{linenos=true, texcl=true}
%\newminted[ocaml1]{Ocaml}{linenos=true, texcl=true, bgcolor=bg}
\usepackage{mathtools}
\usepackage[ruled,vlined]{algorithm2e}

\title{TIPE: Détection de Contours}
\subtitle{Détection de mouvement}
\author{Alban WAXIN n°39534}
\date



\newtcblisting{ocaml3}[1][]{
  listing engine=minted,
  colback=ocamlbg,
  colframe=black!70,
  listing only,
  minted style=monokai,
  minted language=Ocaml,
  minted options={linenos=true,numbersep=3mm,texcl=true,#1},
  left=5mm,enhanced,
  overlay={\begin{tcbclipinterior}\fill[black!25] (frame.south west)
            rectangle ([xshift=5mm]frame.north west);\end{tcbclipinterior}}
}
\definecolor{ocamlbg}{rgb}{0.25,0.25,0.25}

\renewcommand*{\overrightarrow}[1]{\vbox{\halign{##\cr 
  \tiny\rightarrowfill\cr\noalign{\nointerlineskip\vskip1pt} 
  $#1\mskip2mu$\cr}}}

\newcommand*{\Coordp}[4]{% 
  \ensuremath{{#1}\, 
    \left\lvert 
      \begin{matrix} 
        #2\\ 
        #3\\
        #4
      \end{matrix}  
    \right.% Ne pas oublier le délimiteur invisible. 
  }}
  
\newcommand{\divp}[2]
	{
	  \frac{\partial #1}{\partial #2}
	}
	
\newcommand{\fct}[5]
	{
	  \begin{array}{ccccc}
		#1 & : & #2 & \to & #3 \\
	    && #4 & \mapsto & #5 \\
	  \end{array}
	}
	
\newcommand{\divt}[2]
	{
	  \frac{d #1}{d #2}
	}	
	
\newcommand{\divpsnd}[2]
	{
	  \frac{\partial^2 #1}{\partial #2^2}	
	}

\newcommand{\divts}[2]
	{
	  \frac{d^2 #1}{d #2^2}	
	}

\newcommand{\demos}[1]
{
\cy{\underline{\blz Démonstration: }#1}\vspace{0.5cm} 
}

\newcommand{\props}[1]
{
 \begin{mybox}
 \textbf{\rd{\underline{\blz Propriétés:} #1}}
 \vspace{0.5cm}
 \newline
}

\newcommand{\thetap}{\dot{\theta}}

\newcommand{\rot}[1]
{
 \overrightarrow{rot}(\overrightarrow{#1})
}

\newcommand{\grad}[1]
{
 \overrightarrow{grad}(\overrightarrow{#1})
}

\newcommand{\lapl}[1]
{
 \Delta(\overrightarrow{#1})
}

\newcommand{\divs}[1]
{
 div(\overrightarrow{#1})
}

\newcommand{\prope}
{
 \end{mybox}
}

\newcommand{\defis}[1]
{
 \begin{mybox}
 \textbf{\rd{\underline{\blz Définition:} #1}}
 \vspace{0.5cm}
 \newline
}
\newcommand{\defie}
{
 \end{mybox}
}
\newcommand{\discu}[1]
{
 \blue{\underline{\blz Discussion:} #1}
}

\newcommand{\vs}
{
 \vspace{0.25cm}
}
\newcommand{\ns}{\\ \indent \indent \vspace{0.15cm}}
\newcommand{\MP}{\mathbb{P}}
\newcommand{\bxt}[1]{\boxed{\text{#1}}}
\newcommand{\rd}[1]{\textcolor{red}{#1}}
\newcommand{\R}{\mathbb{R}}
\newcommand{\C}{\mathbb{C}}
\newcommand{\N}{\mathbb{N}}
\newcommand{\Q}{\mathbb{Q}}
\newcommand{\Z}{\mathbb{Z}}
\newcommand{\K}{\mathbb{K}}
\newcommand{\M}{\mathcal{M}}

\newcommand{\abs}[1]{\left\lvert#1\right\rvert}
\DeclarePairedDelimiter{\ceil}{\lceil}{\rceil}

\begin{document}
\frame{\titlepage}

\begin{frame}
%\includegraphics{}
\begin{block}
{Problématique:} Comment à partir d'un document image obtenir une répartition exhaustive des différentes constituantes?
\end{block}
\end{frame}

\begin{frame}
\frametitle{Sommaire}
\begin{columns}[t]
  \begin{column}{7.5cm}
  \tableofcontents[sections={1,2}]
  \end{column}
  \begin{column}{4.5cm}
  \tableofcontents[sections={3}]
  \end{column}
  \end{columns}
\setcounter{tocdepth}{2}
\end{frame}

\section{Introduction}
\subsection{Définitions}

\begin{frame}[fragile]
\frametitle{Cadre d'étude}
Détermination des zones d'intérêt d'une image couleur visant au suivi  par expansion ou rétractation de la zone\\

\underline{Définitions}

\begin{itemize}
\item Pixel: triplet comportant l'information de l'identité lumineuse d'une partie de l'image, $(l,u,v) \in \R^3 $
\item Image: Matrice $M \in \mathcal{M}_{n\times m}(\R^3)$ composée d'informations de pixels
\end{itemize}  
\end{frame}

\begin{frame}
\frametitle{Objectif de l'étude}

Dans cette étude l'objectif est double,

\begin{enumerate}
\item A partir d'images considérés séparément, déterminer les différentes zones constituants une séparation de contenu
\item Et cela en utilisant une méthode de smoothing puis de clustering ou une méthode d'expansion

Banques d'étude:
\begin{figure}[h]
    \begin{minipage}[c]{.55\linewidth}
        \centering
        \includegraphics[scale=1]{../Documents/docs LaTeX/Cours de Math/Screenshot_1.jpg}
        \includegraphics[scale=0.90]{../Documents/docs LaTeX/Cours de Math/Screenshot_2.jpg}
        \includegraphics[scale=0.68]{../Documents/docs LaTeX/Cours de Math/Screenshot_3.jpg}
        \caption{Clichés médicaux}
    \end{minipage}
	\begin{minipage}[c]{.39\linewidth}
        \centering
        \includegraphics[scale=0.16]{../Documents/docs LaTeX/Cours de Math/satelitte0.jpg}
        \includegraphics[scale=0.16]{../Documents/docs LaTeX/Cours de Math/satelitte.jpg}
        \caption{Clichés satellites}
	\end{minipage}
\end{figure}
\end{enumerate}
\end{frame}

\section{Homogénéisation}
\subsection{Algorithme du Mean Shift}

\begin{frame}
\frametitle{Partie I}
\begin{block}{}
\begin{center}
Méthode de Smoothing et Clustering
\end{center}
\end{block}
\end{frame}

\begin{frame}
\frametitle{Procédures antérieur}

Avant toute séparation de région, il peut être nécessaire d'uniformiser le contenu.Pour cela on fait appel à un algorithme de "Mean Shift"
\small
\begin{algorithm}[H]
\SetAlgoLined
\KwData{$D$ ensemble de pixels\\
$E$ ensemble de points qui représentent les pics de données\\K une application, représentant géométriquement une fenêtre d'appartenance}
\KwResult{Image homogénéisée}
initialisation\;
\While{Non convergé}{
	M <- listevide\;
	\For{$x \in E$}{
		Placer $K$ en $x$\;
		Déterminer la moyenne $m$ des points de D compris dans la fenêtre\;
		Ajouter $m$ à M
}
	E <- M
}
\caption{Mean Shift}
\end{algorithm}
\normalsize
On suppose $E \subset D$\\
\end{frame}


\begin{frame}
\frametitle{Principe du Mean Shift}
\begin{figure}[h]
    \begin{minipage}[c]{.55\linewidth}
        \centering
        \includegraphics[scale=0.08]{../Documents/docs LaTeX/Cours de Math/explication.jpg}
        \includegraphics[scale=0.08]{../Documents/docs LaTeX/Cours de Math/explication2.jpg}
        \includegraphics[scale=0.08]{../Documents/docs LaTeX/Cours de Math/explication3.jpg}
        \caption{Itérations 0,1,2}
    \end{minipage}
	\begin{minipage}[c]{.39\linewidth}
        \centering
        \includegraphics[scale=0.1]{../Documents/docs LaTeX/Cours de Math/explication4.jpg}
        \includegraphics[scale=0.1]{../Documents/docs LaTeX/Cours de Math/explication5.jpg}
        \caption{Itérations 3,4}
	\end{minipage}
\end{figure}
\end{frame}

\begin{frame}
\frametitle{Représentation 3D}
\begin{figure}
\includegraphics[scale=0.23]{../Documents/docs LaTeX/Cours de Math/principe6.png}
\caption{Espace caractéristique}
\end{figure}
\end{frame}

\subsection{Terminaison de l'algorithme}

\begin{frame}
\frametitle{Initialisation Mathématique}

On muni $\R^3$,de sa norme usuelle et de son produit scalaire <|>\ns

La condition de convergence de l'algorithme est la convergence pour une zone donnée des éléments de E en un seul point 

Ce qui se traduit par un confinement des points de E a une boule de rayon $r$ tendant vers 0.\\


\underline{diamètre:} $d = d(\underbrace{max}_{x \in D}d(x,0),\underbrace{min}_{x \in D}d(x,0))$

or les données ne sont pas réparties uniformément dans $\R^3$ ,\\
Soit $X \in R^3, ||X|| = 1$
On pose alors soit $d_X =  \underbrace{max}_{x \in D}(<x|X>)-\underbrace{min}_{x \in D}(<x|X>)$\\
Et $d = sup_{||X||=1}(d_X)$

\end{frame}

\begin{frame}
Définissons les termes apparaissant dans l'algorithme\\
$\fct{K}{\R^5}{\R}{x}{\begin{cases} f(x), & ||x|| \leq \lambda\\
0, & ||x|| > \lambda
\end{cases}}$  appelée Noyau définit l'appartenance d'un point à un espace\\
Durant l'algorithme on calcule la moyenne:$m(a) = \frac{\sum_{x\in D}K(x-a)x}{\sum_{x\in D}K(x-a)}$\\
On notera $m(D) = \lbrace m(x), x \in D \rbrace$
\end{frame}


\begin{frame}
\frametitle{Convergence de l'algorithme}
Montrons que $d(m(D))< d(D)$\\
Or $d(D) = sup_{||X||=1}(\underbrace{max}_{x \in D}(<x|X>)-\underbrace{min}_{x \in D}(<x|X>))$\\
Soit $X \in \R^3$\\
Notons $\alpha_F = \underbrace{max}_{x \in F}(<x|X>) , \beta_F = \underbrace{min}_{x \in F}(<x|X>)$\\
Montrons alors que $\alpha_{m(D)} - \beta_{m(D)} < \alpha_{D} -\beta_{D}$\\
\begin{align*}
\alpha_{m(D)} - \beta_{m(D)} &\leq \alpha_{D} - \beta_{m(D)} \text{ (linéarité)}\\
&= \alpha_{D} - \beta_{m(D)} + \beta_{D} - \beta_{D}\\
&\leq \alpha_{D}- \beta_{D} + \beta_{D} - \beta_{m(D)}\\
\end{align*}
\end{frame}

\begin{frame}
\frametitle{Convergence de l'algorithme}
Montrons donc que  $\beta_{D} - \beta_{m(D)} < 0 $\\
or $\forall x \in \R^3$
\scriptsize
\begin{align*}
\beta_{D} - <m(x)|X> &= \beta_D - <\frac{\sum_{x' \in D} K(x' - x)x'}{\sum_{x' \in D} K(x' - x)}|X>\\
&= \beta_D - \frac{\sum_{x' \in D} K(x' - x)<x'|X>}{\sum_{x' \in D} K(x' - x)}\\
&= \frac{\sum_{x' \in D} K(x' - x)(\beta_D - <x'|X>)}{\sum_{x' \in D} K(x' - x)}\\
\end{align*}
\normalsize
or $\forall x' \in D, \beta_D - <x'|X> \leq 0$\\
Donc $\frac{\sum_{x' \in D} K(x' - x)(\beta_D - <x'|X>)}{\sum_{x' \in D} K(x' - x)} < 0$\\
Donc $\alpha_{m(D)} - \beta_{m(D)} < \alpha_{D} - \beta_{D}$\\
Donc si cela est vrais pour tout X alors: $d(m(D)) < d(D)$\\
Donc \boxed{\text{L'algorithme termine}} 
\end{frame}

\subsection*{Résultats}

\begin{frame}
\frametitle{Résultats émulé}
Le meanshift n'ayant pas de réalité visuel une estimation du résultat  sur l'image 1 serait
\begin{figure}
\begin{minipage}[c]{.55\linewidth}
        \centering
        \includegraphics[scale=1.5]{../Documents/docs LaTeX/Cours de Math/Résultat 1.jpg}
        \caption{points initiaux}
    \end{minipage}
	\begin{minipage}[c]{.39\linewidth}
        \centering
        \includegraphics[scale=1.5]{../Documents/docs LaTeX/Cours de Math/Résultat 2.jpg}
        \caption{Itérations 3,4}
	\end{minipage}
\end{figure}
\end{frame}

\subsection*{Complexité}

\begin{frame}
\frametitle{Complexité}
Les couts sont:\
\begin{itemize}
\item calcul du noyau: \underline{constant}
\item calcul de la moyenne pour un point : \underline{$n*C$}
\item nombre de calcul de moyenne par itération: \underline{$n$}
\item nombre d'itération : fini , $k$
\end{itemize}
\begin{block}{Total}
total temporel: $k*c*n^2 = O(n^2)$
\end{block} 
\end{frame}

\begin{frame}
\frametitle{Avantages et Défauts}
\begin{figure}
\includegraphics[scale=0.5]{../Documents/docs LaTeX/Cours de Math/avantages.jpg}
\caption{Espace caractéristique}
\end{figure}
\end{frame}

\subsection*{Groupement et Labels}

\begin{frame}
\frametitle{Algorithme}
\scriptsize
\begin{algorithm}[H]
\SetAlgoLined
\KwData{M pixels avec leur "shift"\\
rg Rayon Géométrique, rc Rayon de couleur}
\KwResult{Tableau des labels}
initialisation\;
L <- Tableauvide
\For{$x \in M$}
	{
	\For{$y \in M \to x$}
		{
		\eIf{$x$ Non attribué}
			{
				\eIf{distance < rg et distance couleur < rc}
					{
					Affecter le label de $y$ à $x$
					}
					{\eIf{tableau fini}
						{
						 Créer nouveau Label\\
						 Affecter le label à $x$
						}
						{}
					}
			}
			{}
		}
	}
\caption{Affectation de Labels}
\end{algorithm}
\end{frame}

\begin{frame}
\frametitle{Résultats finaux}
\end{frame}

\section{Croissance de Régions}

\begin{frame}
\begin{block}{Partie II}
Croissance de Régions
\end{block}
\end{frame}

\begin{frame}
\frametitle{Principe}
L'algorithme de Croissance de Régions est un Algorithme en 3 étapes:
\begin{enumerate}
\item Détermination de la "Seed"
\item Croissance connexe des régions
\item Fusion des régions connexe dont la valeur moyenne est assez proche
\end{enumerate} 
\end{frame}

\begin{frame}
\frametitle{Illustration}
\begin{figure}
\includegraphics[scale=0.5]{../Documents/docs LaTeX/Cours de Math/temporaire.png}
\caption{Espace caractéristique}
\end{figure}
\end{frame}

\subsection*{Croissance}

\begin{frame}
\frametitle{Croissance des Régions}
\scriptsize
\begin{algorithm}[H]
\SetAlgoLined
\KwData{S Ensemble de pixel d'intérêts\\
r seuil d'homogénéité}
\KwResult{Matrice des pixels avec leur région}
initialisation\;
Créer une pile P\\
Empiler S dans P\\
\While{P n'est pas vide}{
$x$ <- Dépiler P \\
\eIf{$x$ n'a pas de région}
	{
	V <- Voisins(x)
	\eIf{$\forall v \in$ V, $v$ n'a pas de région}
		{Créer une région et l'affecter à $x$}
		{\eIf{$x$ est homogène à la région}{ajouter la région à\\ Empiler V(sans région) dans P}{Créer une région et l'affecter à $x$}
		}
	}	
	{}
	}
	
\caption{Croissance}
\end{algorithm}
\end{frame}


\begin{frame}
\frametitle{Critère d'homogénéité}
Plusieurs critères sont possibles
\begin{itemize}
\item \underline{Contraste:} P(R) est vrais ssi ${x \in R}(x) -min_{x \in R}(x) \leq r$
\item \underline{Écart-type:} P(R) est vrais ssi $\sum_{x \in R} (x- moy)^2 \leq r^2$
\item \underline{Faiblesse du gradient:} P(R) est vrais ssi $\forall x,y voisins \in R^2, |x-y| < r$
\end{itemize}

Dans une étude en image couleur l'écart-type et la faiblesse du gradient sont plus intéressant car les distances sont calculées par normes (donc applicable à $R^3$)
\end{frame}

\begin{frame}
\frametitle{Complexité}
La partie croissance de l'algorithme parcours au plus l'ensemble des pixels de l'image
donc si tout pixel rentre dans une région ce qui n'est pas forcement le cas la complexité de parcours est de $n^2$ de plus 
\end{frame}



\end{document}

