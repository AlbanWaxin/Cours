\documentclass[12pt]{report}
\usepackage[utf8]{inputenc}
\usepackage[T1]{fontenc}
\usepackage[fleqn]{amsmath}
\usepackage{amsfonts,amssymb,stmaryrd}
\usepackage[english]{babel}
\usepackage{pdfpages}

%=============Affichage=======================
\usepackage{fullpage}
\usepackage{mathtools}
\usepackage{lmodern}
\usepackage{xcolor}
\usepackage{enumitem}
\usepackage{tikz,tkz-tab}
\usepackage[ruled,vlined]{algorithm2e}
\usepackage{booktabs}
\usepackage{setspace}
\usepackage{tikz}
\usetikzlibrary{arrows,automata}
\title{TD Statistiques 3A STI}
\author{Waxin Alban}

\definecolor{almond}{rgb}{0.94, 0.87, 0.8}
\definecolor{champagne}{rgb}{0.97, 0.91, 0.81}
\definecolor{dgreen}{rgb}{0.0, 0.5, 0.0}
\definecolor{bc}{rgb}{0.8588, 0.8980, 0.9450}

\setlength{\topmargin}{-1.5cm}
\setlength{\textheight}{25cm}
\setlength{\textwidth}{16cm}
\setlength{\oddsidemargin}{-1.5cm}
\setlength{\evensidemargin}{50cm}
\doublespacing  

\newcommand{\rd}[1]{\textcolor{red}{#1}}
\newcommand{\g}[1]{\textcolor{lime}{#1}}
\newcommand{\dg}[1]{\textcolor{dgreen}{#1}}
\newcommand{\blue}[1]{\textcolor{blue}{#1}}
\newcommand{\cy}[1]{\textcolor{cyan}{#1}}
\newcommand{\blz}{$\blacklozenge$}
\newcommand{\ns}{\\\indent\indent\vspace{0.25cm}}
\setcounter{secnumdepth}{5}% profondeur de la table des matières
\usepackage{titlesec}


\titleformat{\chapter}[frame]
{\Huge}
{\filright\rmfamily\bfseries\Huge\enspace\thechapter\enspace}
{18pt}
{\rmfamily\huge\bfseries\filcenter}
% rmfamily=roman, sffamily = sans serif ou ttfamily =type writer
\usepackage[many]{tcolorbox} % Creation de box collorable pour le texte non intégré
\newtcolorbox{mybox}{colback=bc,
colframe=black,arc=0mm,sharp corners= northwest,arc=10pt}

\newtcolorbox{demo}{colback=almond,
colframe=black,arc=0mm,sharp corners= northeast,arc=10pt}

\renewcommand*{\overrightarrow}[1]{\vbox{\halign{##\cr
 \tiny\rightarrowfill\cr\noalign{\nointerlineskip\vskip1pt}
 $#1\mskip2mu$\cr}}}

\newcommand{\rem}[1]
{
\subparagraph*{\underline{Remarque:#1}}\mbox{}\\
}

\newcommand{\props}[1]
{
\begin{mybox}
\textbf{\rd{\underline{\blz Propriété:} #1}}
\vspace{0.5cm}
\newline
}

\newcommand{\prope}
{
\end{mybox}
}

\newcommand{\scal}[2]
{
<#1|#2>
}

\newcommand{\defis}[1]
{
\begin{mybox}
\textbf{\rd{\underline{\blz Définition:} #1}}
\vspace{0.5cm}
\newline
}
\newcommand{\defie}
{
\end{mybox}
}
\newcommand{\demos}[1]
{
\begin{demo}
\textbf{\underline{\blz Démonstration:} #1}
\newline
}
\newcommand{\demoe}
{
\end{demo}
}
\newcommand{\exe}[1]
{
\subparagraph*{\underline{Exemple:#1}}\mbox{}\\
}

\newcommand{\vs}
{
\vspace{0.25cm}
}

\newcommand{\thms}[1]
{
\begin{mybox}
\textbf{\rd{\underline{\blz Théorème:} #1}}
\vspace{0.5cm}
\newline
}

\newcommand{\thme}
{
\end{mybox}
}

\newcommand{\coros}[1]
{
\begin{mybox}
\textbf{\rd{\underline{\blz Corolaire:} #1}}
\vspace{0.5cm}
\newline
}

\newcommand{\coroe}
{
\end{mybox}
}

\newcommand{\lems}[1]
{
\begin{mybox}
\textbf{\rd{\underline{\blz Lemme:} #1}}
\vspace{0.5cm}
\newline
}

\newcommand{\leme}
{
\end{mybox}
}
%=============================================

%\usepackage[cm]{aeguill}

%=============Mathématiques=================

%--------------Raccourcis:------------------
\newcommand{\R}{\mathbb{R}}
\newcommand{\C}{\mathbb{C}}
\newcommand{\N}{\mathbb{N}}
\newcommand{\Q}{\mathbb{Q}}
\newcommand{\Z}{\mathbb{Z}}
\newcommand{\K}{\mathbb{K}}
\newcommand{\M}{\mathcal{M}}
\newcommand{\nint}[1]{#1 \in \N}
\newcommand{\zint}[1]{#1 \in \N^*}
\newcommand{\limi}[1]{\underset{#1 \to \infty}{lim}}
\newcommand{\limn}[2]{\underset{#1 \to #2}{lim}}
\newcommand{\x}{\times}
\newcommand{\un}[1]{u_{#1}}
\newcommand{\uns}{(u_n)_\nint{n}}
\newcommand{\Sn}[1]{S_{#1}}
\newcommand{\Sns}{(S_n)_\nint{n}}
\newcommand{\ol}[1]{\overline{#1}}
\newcommand{\znz}{\Z/n\Z}
\renewcommand{\o}{\circ}

\newcommand{\seriegu}{\sum u_n}
\newcommand{\seriegv}{\sum v_n}
\newcommand{\harmonique}{\sum \frac{1}{n}}
\newcommand{\SRieman}{\sum \frac{1}{n^\alpha}}
\newcommand{\serie}[3]{\sum_{#1}^{#2}{#3}}
\newcommand{\satps}{série à terme positif}
\newcommand{\satp}{séries à termes positifs}
\newcommand{\pl}[1]{\mathbb{#1}[X]}
\newcommand{\som}[2]{\sum\limits_{#1}^{#2}}


\newcommand{\abs}[1]{\left\lvert#1\right\rvert}
\DeclarePairedDelimiter{\ceil}{\lceil}{\rceil}

%Format de fonctions:
\newcommand{\fct}[5]
	{
	  \begin{array}{ccccc}
		#1 & : & #2 & \to & #3 \\
	    && #4 & \mapsto & #5 \\
	  \end{array}
	}
\newcommand{\dfct}[2] {#1 \mapsto #2}

\renewcommand{\abs}[1]{|#1|}
\newcommand{\nm}[2]{ ||#1||_{#2} }
%===================TESTS===================

\begin{document}
\chapter{Langages Formels}
\section{Exercice 1}
\section{Exercice 2}
\section{Exercice 3}
\begin{itemize}
    \item \begin{align*}
        \begin{cases}
            A = ( 01^* + 1) A +B\\
            B = 11 + 1A +00C\\
            C = \varepsilon + A +B
        \end{cases} &\Leftrightarrow 
        \begin{cases}
            A = ( 01^* + 1)^*B = \Sigma^* B\\
            B = 11 + 1\Sigma^* B +00 + 00\Sigma^* B + 00B\\
            C = \varepsilon + (00+1)\Sigma^* B = ((00+1)\Sigma^*)^* (00+11) = ((00+1)\Sigma^*)^*(00+11)
        \end{cases}\\
        & \Leftrightarrow
        \begin{cases}
            A = ( 01^* + 1)^*B = \Sigma^* B\\
            B = 11 + 1\Sigma^* B +00 + 00\Sigma^* B + 00B\\
            C = \varepsilon + \Sigma^* (00 + 11) + (00+11) = \Sigma^*(00+11)+\varepsilon
        \end{cases}
        \end{align*} 
    \item DEMO ARDEN     
\end{itemize}

\chapter{Expressions rationnelles}
\section{Exercice 4 et 5}

\begin{itemize}
    \item $(a+b)^*(aa + bb)(a+b)^*$\\
    \begin{center}
        \begin{tikzpicture}[scale=0.2]
        \tikzstyle{every node}+=[inner sep=0pt]
        \draw [black] (11.6,-13.5) circle (3);
        \draw (11.6,-13.5) node {$q_0$};
        \draw [black] (31.1,-5) circle (3);
        \draw (31.1,-5) node {$q_1$};
        \draw [black] (31.1,-22.7) circle (3);
        \draw (31.1,-22.7) node {$q_2$};
        \draw [black] (50.5,-13.5) circle (3);
        \draw (50.5,-13.5) node {$q_3$};
        \draw [black] (8.617,-13.319) arc (294.25512:6.25512:2.25);
        \draw (5.08,-8.93) node [left] {$a,b$};
        \fill [black] (9.93,-11.02) -- (10.06,-10.09) -- (9.15,-10.5);
        \draw [black] (14.35,-12.3) -- (28.35,-6.2);
        \fill [black] (28.35,-6.2) -- (27.42,-6.06) -- (27.82,-6.98);
        \draw (22.42,-9.76) node [below] {$a$};
        \draw [black] (33.85,-6.2) -- (47.75,-12.3);
        \fill [black] (47.75,-12.3) -- (47.22,-11.52) -- (46.82,-12.43);
        \draw (39.73,-9.76) node [below] {$a$};
        \draw [black] (50.681,-10.517) arc (204.25512:-83.74488:2.25);
        \draw (55.9,-7.14) node [above] {$a,b$};
        \fill [black] (52.98,-11.83) -- (53.91,-11.96) -- (53.5,-11.05);
        \draw [black] (14.31,-14.78) -- (28.39,-21.42);
        \fill [black] (28.39,-21.42) -- (27.88,-20.63) -- (27.45,-21.53);
        \draw (20.23,-18.61) node [below] {$b$};
        \draw [black] (33.81,-21.41) -- (47.79,-14.79);
        \fill [black] (47.79,-14.79) -- (46.85,-14.68) -- (47.28,-15.58);
        \draw (41.92,-18.61) node [below] {$b$};
        \end{tikzpicture}
        \begin{tikzpicture}[scale=0.25]
            \tikzstyle{every node}+=[inner sep=0pt]
            \draw [black] (9.1,-12.8) circle (3);
            \draw (9.1,-12.8) node {$0$};
            \draw [black] (29.7,-5.2) circle (3);
            \draw (29.7,-5.2) node {$0,1$};
            \draw [black] (29.7,-20.3) circle (3);
            \draw (29.7,-20.3) node {$0,2$};
            \draw [black] (53.5,-5.2) circle (3);
            \draw (53.5,-5.2) node {$0,1,3$};
            \draw [black] (53.5,-5.2) circle (2.4);
            \draw [black] (53.5,-20.3) circle (3);
            \draw (53.5,-20.3) node {$0,2,3$};
            \draw [black] (53.5,-20.3) circle (2.4);
            \draw [black] (11.91,-11.76) -- (26.89,-6.24);
            \fill [black] (26.89,-6.24) -- (25.96,-6.05) -- (26.31,-6.98);
            \draw (20.42,-9.53) node [below] {$a$};
            \draw [black] (11.92,-13.83) -- (26.88,-19.27);
            \fill [black] (26.88,-19.27) -- (26.3,-18.53) -- (25.96,-19.47);
            \draw (18.34,-17.08) node [below] {$b$};
            \draw [black] (32.7,-20.3) -- (50.5,-20.3);
            \fill [black] (50.5,-20.3) -- (49.7,-19.8) -- (49.7,-20.8);
            \draw (41.6,-20.8) node [below] {$b$};
            \draw [black] (32.7,-5.2) -- (50.5,-5.2);
            \fill [black] (50.5,-5.2) -- (49.7,-4.7) -- (49.7,-5.7);
            \draw (41.6,-5.7) node [below] {$a$};
            \draw [black] (29.7,-17.3) -- (29.7,-8.2);
            \fill [black] (29.7,-8.2) -- (29.2,-9) -- (30.2,-9);
            \draw (30.2,-12.75) node [right] {$a$};
            \draw [black] (29.7,-8.2) -- (29.7,-17.3);
            \fill [black] (29.7,-17.3) -- (30.2,-16.5) -- (29.2,-16.5);
            \draw (29.2,-12.75) node [left] {$b$};
            \draw [black] (53.5,-8.2) -- (53.5,-17.3);
            \fill [black] (53.5,-17.3) -- (54,-16.5) -- (53,-16.5);
            \draw (53,-12.75) node [left] {$b$};
            \draw [black] (53.5,-17.3) -- (53.5,-8.2);
            \fill [black] (53.5,-8.2) -- (53,-9) -- (54,-9);
            \draw (54,-12.75) node [right] {$a$};
            \draw [black] (55.6,-3.074) arc (163.09349:-124.90651:2.25);
            \draw (60.64,-2.25) node [right] {$a$};
            \fill [black] (56.46,-5.57) -- (57.08,-6.28) -- (57.38,-5.33);
            \draw [black] (56.463,-20.686) arc (110.30993:-177.69007:2.25);
            \draw (59.73,-25.38) node [right] {$b$};
            \fill [black] (55,-22.89) -- (54.8,-23.81) -- (55.74,-23.46);
        \end{tikzpicture}
        \begin{tikzpicture}[scale=0.2]
            \tikzstyle{every node}+=[inner sep=0pt]
            \draw [black] (9.4,-12) circle (3);
            \draw (9.4,-12) node {$I$};
            \draw [black] (27,-4.8) circle (3);
            \draw (27,-4.8) node {$II$};
            \draw [black] (26.9,-19.6) circle (3);
            \draw (26.9,-19.6) node {$III$};
            \draw [black] (49,-12) circle (3);
            \draw (49,-12) node {$IV$};
            \draw [black] (49,-12) circle (2.4);
            \draw [black] (12.18,-10.86) -- (24.22,-5.94);
            \fill [black] (24.22,-5.94) -- (23.29,-5.78) -- (23.67,-6.7);
            \draw (19.25,-8.92) node [below] {$a$};
            \draw [black] (51.68,-10.677) arc (144:-144:2.25);
            \draw (56.25,-12) node [right] {$a,b$};
            \fill [black] (51.68,-13.32) -- (52.03,-14.2) -- (52.62,-13.39);
            \draw [black] (29.85,-5.73) -- (46.15,-11.07);
            \fill [black] (46.15,-11.07) -- (45.54,-10.34) -- (45.23,-11.29);
            \draw (37.03,-8.94) node [below] {$a$};
            \draw [black] (12.15,-13.2) -- (24.15,-18.4);
            \fill [black] (24.15,-18.4) -- (23.61,-17.63) -- (23.22,-18.54);
            \draw (17.04,-16.31) node [below] {$b$};
            \draw [black] (29.74,-18.62) -- (46.16,-12.98);
            \fill [black] (46.16,-12.98) -- (45.24,-12.76) -- (45.57,-13.71);
            \draw (38.98,-16.34) node [below] {$b$};
            \draw [black] (26.92,-16.6) -- (26.98,-7.8);
            \fill [black] (26.98,-7.8) -- (26.47,-8.6) -- (27.47,-8.6);
            \draw (27.46,-12.2) node [right] {$a$};
            \draw [black] (26.98,-7.8) -- (26.92,-16.6);
            \fill [black] (26.92,-16.6) -- (27.43,-15.8) -- (26.43,-15.8);
            \draw (26.44,-12.2) node [left] {$b$};
        \end{tikzpicture}
            
    \end{center}   
    \item $((a+b)^3)^*$
\end{itemize}

 

\end{document}